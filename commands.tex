
\usepackage{tikzit}
\input{tikz/zxw_styles.tikzstyles}

\newtheorem{lemma}{Lemma}[section]
\newtheorem{thm}{Theorem}[section]
\theoremstyle{definition}
\newtheorem{definition}{Definition}[section]
\newtheorem{prop}{Proposition}
\newtheorem*{prop*}{Proposition}
\newtheorem{corollary}{Corollary}
\newtheorem{remark}{Remark}

%custom zx commands
\newcommand{\redzero}{\begin{tikzpicture}
	\begin{pgfonlayer}{nodelayer}
		\node [style=rn] (0) at (0, 0) {};
		\node [style=none] (1) at (0, -0.5) {};
	\end{pgfonlayer}
	\begin{pgfonlayer}{edgelayer}
		\draw (0) to (1.center);
	\end{pgfonlayer}
\end{tikzpicture}
}

\newcommand{\redpi}{\begin{tikzpicture}
	\begin{pgfonlayer}{nodelayer}
		\node [style={rn_phase}] (0) at (0, 0) {$\pi$};
		\node [style=none] (1) at (0, -0.5) {};
	\end{pgfonlayer}
	\begin{pgfonlayer}{edgelayer}
		\draw (0) to (1.center);
	\end{pgfonlayer}
\end{tikzpicture}
}

\newcommand{\redzeroup}{\begin{tikzpicture}
	\begin{pgfonlayer}{nodelayer}
		\node [style={rn}] (0) at (0, 0) {};
		\node [style=none] (1) at (0, 0.5) {};
	\end{pgfonlayer}
	\begin{pgfonlayer}{edgelayer}
		\draw (0) to (1.center);
	\end{pgfonlayer}
\end{tikzpicture}
}

\newcommand{\redpiup}{\begin{tikzpicture}
	\begin{pgfonlayer}{nodelayer}
		\node [style={rn_phase}] (0) at (0, 0) {$\pi$};
		\node [style=none] (1) at (0, 0.5) {};
	\end{pgfonlayer}
	\begin{pgfonlayer}{edgelayer}
		\draw (0) to (1.center);
	\end{pgfonlayer}
\end{tikzpicture}
}

\newcommand{\greenzero}{\begin{tikzpicture}
	\begin{pgfonlayer}{nodelayer}
		\node [style=gn] (0) at (0, 0) {};
		\node [style=none] (1) at (0, -0.5) {};
	\end{pgfonlayer}
	\begin{pgfonlayer}{edgelayer}
		\draw (0) to (1.center);
	\end{pgfonlayer}
\end{tikzpicture}
}

\newcommand{\greenpi}{\begin{tikzpicture}
	\begin{pgfonlayer}{nodelayer}
		\node [style={gn_phase}] (0) at (0, 0) {$\pi$};
		\node [style=none] (1) at (0, -0.5) {};
	\end{pgfonlayer}
	\begin{pgfonlayer}{edgelayer}
		\draw (0) to (1.center);
	\end{pgfonlayer}
\end{tikzpicture}
}

\newcommand{\greenzeroup}{\begin{tikzpicture}
	\begin{pgfonlayer}{nodelayer}
		\node [style={gn}] (0) at (0, 0) {};
		\node [style=none] (1) at (0, 0.5) {};
	\end{pgfonlayer}
	\begin{pgfonlayer}{edgelayer}
		\draw (0) to (1.center);
	\end{pgfonlayer}
\end{tikzpicture}
}

\newcommand{\ghzstate}[1][0.5]{\begin{tikzpicture}[scale=#1]
	\begin{pgfonlayer}{nodelayer}
		\node [style=gn] (0) at (0, 0) {};
		\node [style=none] (1) at (-1, -1) {};
		\node [style=none] (2) at (1, -1) {};
		\node [style=none] (3) at (0, -1) {};
	\end{pgfonlayer}
	\begin{pgfonlayer}{edgelayer}
		\draw [bend right] (0) to (1.center);
		\draw [bend left] (0) to (2.center);
		\draw (0) to (3.center);
	\end{pgfonlayer}
\end{tikzpicture}}

\newcommand{\wstate}[1][0.5]{\begin{tikzpicture}[scale=#1]
	\begin{pgfonlayer}{nodelayer}
		\node [style=uw] (0) at (0, 0) {};
		\node [style=none] (1) at (-1, -1) {};
		\node [style=none] (2) at (1, -1) {};
		\node [style=none] (3) at (0, -1) {};
	\end{pgfonlayer}
	\begin{pgfonlayer}{edgelayer}
		\draw [bend right] (0) to (1.center);
		\draw [bend left] (0) to (2.center);
		\draw (0) to (3.center);
	\end{pgfonlayer}
\end{tikzpicture}}

\newcommand{\zspid}[1][0.5]{\begin{tikzpicture}[scale=#1]
	\begin{pgfonlayer}{nodelayer}
		\node [style=gn] (0) at (0, 0) {};
		\node [style=none] (1) at (0.5, -0.5) {};
		\node [style=none] (2) at (-0.5, -0.5) {};
		\node [style=none] (3) at (0, 0.75) {};
	\end{pgfonlayer}
	\begin{pgfonlayer}{edgelayer}
		\draw (0) to (1.center);
		\draw (2.center) to (0);
		\draw (3.center) to (0);
	\end{pgfonlayer}
\end{tikzpicture}
}

\newcommand{\xspid}[1][0.5]{\begin{tikzpicture}[scale=#1]
	\begin{pgfonlayer}{nodelayer}
		\node [style=rn] (0) at (0, 0) {};
		\node [style=none] (1) at (0.5, -0.5) {};
		\node [style=none] (2) at (-0.5, -0.5) {};
		\node [style=none] (3) at (0, 0.75) {};
	\end{pgfonlayer}
	\begin{pgfonlayer}{edgelayer}
		\draw (0) to (1.center);
		\draw (2.center) to (0);
		\draw (3.center) to (0);
	\end{pgfonlayer}
\end{tikzpicture}
}


\newcommand{\coZ}[1][0.5]{\begin{tikzpicture}[scale=#1]
	\begin{pgfonlayer}{nodelayer}
		\node [style=gn] (0) at (0, 0) {};
		\node [style=none] (1) at (0.5, 0.5) {};
		\node [style=none] (2) at (-0.5, 0.5) {};
		\node [style=none] (3) at (0, -0.75) {};
	\end{pgfonlayer}
	\begin{pgfonlayer}{edgelayer}
		\draw (0) to (1.center);
		\draw (2.center) to (0);
		\draw (3.center) to (0);
	\end{pgfonlayer}
\end{tikzpicture}
}


\newcommand{\coX}[1][0.5]{\begin{tikzpicture}[scale=#1]
	\begin{pgfonlayer}{nodelayer}
		\node [style=rn] (0) at (0, 0) {};
		\node [style=none] (1) at (0.5, 0.5) {};
		\node [style=none] (2) at (-0.5, 0.5) {};
		\node [style=none] (3) at (0, -0.75) {};
	\end{pgfonlayer}
	\begin{pgfonlayer}{edgelayer}
		\draw (0) to (1.center);
		\draw (2.center) to (0);
		\draw (3.center) to (0);
	\end{pgfonlayer}
\end{tikzpicture}
}


\newcommand{\wspid}[1][0.5]{\begin{tikzpicture}[scale=#1]
	\begin{pgfonlayer}{nodelayer}
		\node [style=uw] (0) at (0, 0) {};
		\node [style=none] (1) at (0.5, -0.5) {};
		\node [style=none] (2) at (-0.5, -0.5) {};
		\node [style=none] (3) at (0, 0.75) {};
	\end{pgfonlayer}
	\begin{pgfonlayer}{edgelayer}
		\draw (0) to (1.center);
		\draw (2.center) to (0);
		\draw (3.center) to (0);
	\end{pgfonlayer}
\end{tikzpicture}
}

\newcommand{\coW}[1][0.4]{\begin{tikzpicture}[scale=#1]
	\begin{pgfonlayer}{nodelayer}
		\node [style=dw] (0) at (0, 0) {};
		\node [style=none] (1) at (0.5, 0.5) {};
		\node [style=none] (2) at (-0.5, 0.5) {};
		\node [style=none] (3) at (0, -0.75) {};
	\end{pgfonlayer}
	\begin{pgfonlayer}{edgelayer}
		\draw (0) to (1.center);
		\draw (2.center) to (0);
		\draw (3.center) to (0);
	\end{pgfonlayer}
\end{tikzpicture}
}


%
% same but with dots
%

\newcommand{\zspids}[1][0.5]{\begin{tikzpicture}[scale=#1]
	\begin{pgfonlayer}{nodelayer}
		\node [style=gn] (0) at (0, 0) {};
		\node [style=none] (1) at (0.5, -0.5) {};
		\node [style=none] (2) at (-0.5, -0.5) {};
		\node [style=none] (3) at (0, 0.75) {};
            \node [style=label] (4) at (0, -0.5) {$\ldots$};
	\end{pgfonlayer}
	\begin{pgfonlayer}{edgelayer}
		\draw (0) to (1.center);
		\draw (2.center) to (0);
		\draw (3.center) to (0);
	\end{pgfonlayer}
\end{tikzpicture}
}


\newcommand{\xspids}[1][0.5]{\begin{tikzpicture}[scale=#1]
	\begin{pgfonlayer}{nodelayer}
		\node [style=rn] (0) at (0, 0) {};
		\node [style=none] (1) at (0.5, -0.5) {};
		\node [style=none] (2) at (-0.5, -0.5) {};
		\node [style=none] (3) at (0, 0.75) {};
            \node [style=label] (4) at (0, -0.5) {$\ldots$};
	\end{pgfonlayer}
	\begin{pgfonlayer}{edgelayer}
		\draw (0) to (1.center);
		\draw (2.center) to (0);
		\draw (3.center) to (0);
	\end{pgfonlayer}
\end{tikzpicture}
}

\newcommand{\wspids}[1][0.5]{\begin{tikzpicture}[scale=#1]
	\begin{pgfonlayer}{nodelayer}
		\node [style=uw] (0) at (0, 0) {};
		\node [style=none] (1) at (0.5, -0.5) {};
		\node [style=none] (2) at (-0.5, -0.5) {};
		\node [style=none] (3) at (0, 0.75) {};
            \node [style=label] (4) at (0, -0.5) {$\ldots$};
	\end{pgfonlayer}
	\begin{pgfonlayer}{edgelayer}
		\draw (0) to (1.center);
		\draw (2.center) to (0);
		\draw (3.center) to (0);
	\end{pgfonlayer}
\end{tikzpicture}
}

\newcommand{\coZs}[1][0.5]{\begin{tikzpicture}[scale=#1]
	\begin{pgfonlayer}{nodelayer}
		\node [style=gn] (0) at (0, 0) {};
		\node [style=none] (1) at (0.5, 0.5) {};
		\node [style=none] (2) at (-0.5, 0.5) {};
		\node [style=none] (3) at (0, -0.75) {};
            \node [style=label] (4) at (0, 0.6) {$\ldots$};
	\end{pgfonlayer}
	\begin{pgfonlayer}{edgelayer}
		\draw (0) to (1.center);
		\draw (2.center) to (0);
		\draw (3.center) to (0);
	\end{pgfonlayer}
\end{tikzpicture}
}

\newcommand{\coXs}[1][0.5]{\begin{tikzpicture}[scale=#1]
	\begin{pgfonlayer}{nodelayer}
		\node [style=rn] (0) at (0, 0) {};
		\node [style=none] (1) at (0.5, 0.5) {};
		\node [style=none] (2) at (-0.5, 0.5) {};
		\node [style=none] (3) at (0, -0.75) {};
            \node [style=label] (4) at (0, 0.6) {$\ldots$};
	\end{pgfonlayer}
	\begin{pgfonlayer}{edgelayer}
		\draw (0) to (1.center);
		\draw (2.center) to (0);
		\draw (3.center) to (0);
	\end{pgfonlayer}
\end{tikzpicture}
}

\newcommand{\coWs}[1][0.5]{\begin{tikzpicture}[scale=#1]
	\begin{pgfonlayer}{nodelayer}
		\node [style=dw] (0) at (0, 0) {};
		\node [style=none] (1) at (0.5, 0.5) {};
		\node [style=none] (2) at (-0.5, 0.5) {};
		\node [style=none] (3) at (0, -0.75) {};
		\node [style=label] (4) at (0, 0.6) {$\ldots$};
	\end{pgfonlayer}
	\begin{pgfonlayer}{edgelayer}
		\draw (0) to (1.center);
		\draw (2.center) to (0);
		\draw (3.center) to (0);
	\end{pgfonlayer}
\end{tikzpicture}}






\newcommand{\zgate}[2][0.5]{\begin{tikzpicture}[scale=#1]
	\begin{pgfonlayer}{nodelayer}
		\node [style={gn_phase}] (0) at (0, 0) {$#2$};
		\node [style=none] (3) at (0, 0.75) {};
		\node [style=none] (4) at (0, -0.75) {};
	\end{pgfonlayer}
	\begin{pgfonlayer}{edgelayer}
		\draw (3.center) to (0);
		\draw (0) to (4.center);
	\end{pgfonlayer}
\end{tikzpicture}
}

\newcommand{\xgate}[1][0.5]{\begin{tikzpicture}[scale=#1]
	\begin{pgfonlayer}{nodelayer}
		\node [style={rn_phase}] (0) at (0, 0) {$\pi$};
		\node [style=none] (3) at (0, 0.75) {};
		\node [style=none] (4) at (0, -0.75) {};
	\end{pgfonlayer}
	\begin{pgfonlayer}{edgelayer}
		\draw (3.center) to (0);
		\draw (0) to (4.center);
	\end{pgfonlayer}
\end{tikzpicture}
}


\newcommand{\hgate}[1][0.5]{\begin{tikzpicture}[scale=#1]
	\begin{pgfonlayer}{nodelayer}
		\node [style=hbox] (0) at (0, 0) {};
		\node [style=none] (1) at (0, 0.75) {};
		\node [style=none] (2) at (0, -0.75) {};
	\end{pgfonlayer}
	\begin{pgfonlayer}{edgelayer}
		\draw (2.center) to (0);
		\draw (0) to (1.center);
	\end{pgfonlayer}
\end{tikzpicture}
}

\newcommand{\numbergate}[1][a]{\begin{tikzpicture}[scale=0.5]
	\begin{pgfonlayer}{nodelayer}
		\node [style=gbox] (0) at (0, 0) {$#1$};
		\node [style=none] (3) at (0, 0.75) {};
		\node [style=none] (4) at (0, -0.75) {};
	\end{pgfonlayer}
	\begin{pgfonlayer}{edgelayer}
		\draw (3.center) to (0);
		\draw (0) to (4.center);
	\end{pgfonlayer}
\end{tikzpicture}}

\newcommand{\zeroproj}[1][0.5]{\begin{tikzpicture}[scale=#1]
	\begin{pgfonlayer}{nodelayer}
		\node [style=rn] (0) at (0, 0) {};
		\node [style=none] (3) at (0, 0.75) {};
		\node [style=rn] (4) at (0.75, -0.75) {};
		\node [style=none] (5) at (0.75, -1.5) {};
		\node [style=label] (6) at (0, -1) {$\ldots$};
		\node [style=rn] (7) at (-0.75, -0.75) {};
		\node [style=none] (8) at (-0.75, -1.5) {};
	\end{pgfonlayer}
	\begin{pgfonlayer}{edgelayer}
		\draw (3.center) to (0);
		\draw (5.center) to (4);
		\draw (8.center) to (7);
	\end{pgfonlayer}
\end{tikzpicture}
}

\newcommand{\zeroprojplus}[1][0.5]{\begin{tikzpicture}[scale=#1]
	\begin{pgfonlayer}{nodelayer}
		\node [style=gn] (0) at (0, 0) {};
		\node [style=none] (3) at (0, 0.75) {};
		\node [style=rn] (4) at (0.75, -0.75) {};
		\node [style=none] (5) at (0.75, -1.5) {};
		\node [style=label] (6) at (0, -1) {$\ldots$};
		\node [style=rn] (7) at (-0.75, -0.75) {};
		\node [style=none] (8) at (-0.75, -1.5) {};
	\end{pgfonlayer}
	\begin{pgfonlayer}{edgelayer}
		\draw (3.center) to (0);
		\draw (5.center) to (4);
		\draw (8.center) to (7);
	\end{pgfonlayer}
\end{tikzpicture}
}

\newcommand{\cpsi}[1][0.5]{\begin{tikzpicture}[scale=#1]
	\begin{pgfonlayer}{nodelayer}
		\node [style=none] (0) at (0, 1.5) {};
		\node [style=none] (1) at (0, 0.75) {};
		\node [style=none] (2) at (-1, -0.25) {};
		\node [style=none] (3) at (1, -0.25) {};
		\node [style=none] (4) at (-0.5, -0.25) {};
		\node [style=none] (5) at (0.5, -0.25) {};
		\node [style=label] (6) at (0, 0) {$\tilde{\psi}$};
		\node [style=none] (7) at (-0.5, -1) {};
		\node [style=none] (8) at (0.5, -1) {};
	\end{pgfonlayer}
	\begin{pgfonlayer}{edgelayer}
		\draw (2.center) to (1.center);
		\draw (2.center) to (3.center);
		\draw (3.center) to (1.center);
		\draw (0.center) to (1.center);
		\draw (4.center) to (7.center);
		\draw (5.center) to (8.center);
	\end{pgfonlayer}
\end{tikzpicture}
}

\newcommand{\numberstate}[1][a]{\begin{tikzpicture}[scale=0.5]
	\begin{pgfonlayer}{nodelayer}
		\node [style=gbox] (0) at (0, 0) {$#1$};
		\node [style=none] (1) at (0, 1) {};
	\end{pgfonlayer}
	\begin{pgfonlayer}{edgelayer}
		\draw (1.center) to (0);
	\end{pgfonlayer}
\end{tikzpicture}
}

\newcommand{\emptydiag}[1][0.4]{\begin{tikzpicture}[scale=#1]
	\begin{pgfonlayer}{nodelayer}
		\node [style=none] (0) at (0.75, -1.5) {$\cdot$};
		\node [style=none] (1) at (-0.25, 0) {$\cdot$};
		\node [style=none] (2) at (0.75, 0) {$\cdot$};
		\node [style=none] (3) at (-0.75, 0) {$\cdot$};
		\node [style=none] (4) at (0.75, -0.5) {$\cdot$};
		\node [style=none] (5) at (0.25, 0) {$\cdot$};
		\node [style=none] (6) at (-0.75, -0.5) {$\cdot$};
		\node [style=none] (7) at (-0.25, -1.5) {$\cdot$};
		\node [style=none] (8) at (0.75, -1) {$\cdot$};
		\node [style=none] (9) at (-0.75, -1.5) {$\cdot$};
		\node [style=none] (10) at (-0.75, -1) {$\cdot$};
		\node [style=none] (11) at (0.25, -1.5) {$\cdot$};
	\end{pgfonlayer}
\end{tikzpicture}}

\newcommand{\idwire}[1][0.5]{\begin{tikzpicture}[scale=#1]
	\begin{pgfonlayer}{nodelayer}
		\node [style=none] (0) at (0, 0) {};
		\node [style=none] (1) at (0, -1) {};
	\end{pgfonlayer}
	\begin{pgfonlayer}{edgelayer}
		\draw (0.center) to (1.center);
	\end{pgfonlayer}
\end{tikzpicture}
}

\newcommand{\swap}[1][0.25]{\begin{tikzpicture}[scale=#1]
	\begin{pgfonlayer}{nodelayer}
		\node [style=none] (0) at (-1, -1) {};
		\node [style=none] (1) at (1, 1) {};
		\node [style=none] (2) at (1, -1) {};
		\node [style=none] (3) at (-1, 1) {};
	\end{pgfonlayer}
	\begin{pgfonlayer}{edgelayer}
		\draw [in=90, out=-90, looseness=0.75] (3.center) to (2.center);
		\draw [in=-90, out=90, looseness=0.75] (0.center) to (1.center);
	\end{pgfonlayer}
\end{tikzpicture}}

\newcommand{\ccap}[1][0.5]{\begin{tikzpicture}[scale=#1]
	\begin{pgfonlayer}{nodelayer}
		\node [style=none] (0) at (0, 0) {};
		\node [style=none] (1) at (1, 0) {};
	\end{pgfonlayer}
	\begin{pgfonlayer}{edgelayer}
		\draw [bend left=90, looseness=1.75] (0.center) to (1.center);
	\end{pgfonlayer}
\end{tikzpicture}}

\newcommand{\ccup}[1][0.5]{\begin{tikzpicture}[scale=#1]
	\begin{pgfonlayer}{nodelayer}
		\node [style=none] (0) at (0, 0) {};
		\node [style=none] (1) at (1, 0) {};
	\end{pgfonlayer}
	\begin{pgfonlayer}{edgelayer}
		\draw [bend right=90, looseness=1.75] (0.center) to (1.center);
	\end{pgfonlayer}
\end{tikzpicture}}

\newcommand{\redcup}[1][0.4]{\begin{tikzpicture}[scale=#1]
	\begin{pgfonlayer}{nodelayer}
		\node [style=none] (0) at (-1, 1) {};
		\node [style={rn_phase}] (1) at (0, 0) {$\pi$};
		\node [style=none] (2) at (1, 1) {};
	\end{pgfonlayer}
	\begin{pgfonlayer}{edgelayer}
		\draw [bend right] (0.center) to (1);
		\draw [bend right] (1) to (2.center);
	\end{pgfonlayer}
\end{tikzpicture}
}

\newcommand{\ddiag}[1][0.5]{\begin{tikzpicture}[scale=#1]
	\begin{pgfonlayer}{nodelayer}
		\node [style=none] (0) at (-1, 0.5) {};
		\node [style=none] (1) at (1, 0.5) {};
		\node [style=none] (2) at (-1, -0.5) {};
		\node [style=none] (3) at (1, -0.5) {};
		\node [style=none] (4) at (0, 0.5) {};
		\node [style=none] (5) at (-0.5, -0.5) {};
		\node [style=none] (6) at (0.5, -0.5) {};
		\node [style=none] (7) at (0, 0) {$D$};
		\node [style=none] (8) at (-0.5, -1.5) {};
		\node [style=none] (9) at (0, 1.5) {};
		\node [style=none] (10) at (0.5, -1.5) {};
		\node [style=label] (11) at (0, -1.25) {$\ldots$};
	\end{pgfonlayer}
	\begin{pgfonlayer}{edgelayer}
		\draw (0.center) to (1.center);
		\draw (1.center) to (3.center);
		\draw (3.center) to (2.center);
		\draw (2.center) to (0.center);
		\draw (5.center) to (8.center);
		\draw (6.center) to (10.center);
		\draw (4.center) to (9.center);
	\end{pgfonlayer}
\end{tikzpicture}}

\newcommand{\xsq}[1][0.4]{\begin{tikzpicture}[scale=#1]
	\begin{pgfonlayer}{nodelayer}
		\node [style=none] (0) at (0, -1.5) {};
		\node [style=dw] (1) at (0, -0.75) {};
		\node [style=gn] (2) at (0, 0.5) {};
		\node [style=none] (3) at (0, 1.25) {};
	\end{pgfonlayer}
	\begin{pgfonlayer}{edgelayer}
		\draw (3.center) to (2);
		\draw [bend left] (2) to (1);
		\draw [bend right] (2) to (1);
		\draw (1) to (0.center);
	\end{pgfonlayer}
\end{tikzpicture}
}

\newcommand{\hopf}[1][0.4]{\begin{tikzpicture}[scale=#1]
	\begin{pgfonlayer}{nodelayer}
		\node [style=gn] (0) at (0, 0.5) {};
		\node [style=rn] (1) at (0, -0.5) {};
		\node [style=none] (2) at (0, -1.25) {};
		\node [style=none] (3) at (0, 1.25) {};
	\end{pgfonlayer}
	\begin{pgfonlayer}{edgelayer}
		\draw [bend right=45] (0) to (1);
		\draw [bend left=45] (0) to (1);
		\draw (3.center) to (0);
		\draw (1) to (2.center);
	\end{pgfonlayer}
\end{tikzpicture}
}

\newcommand{\arithgate}[1]{\begin{tikzpicture}[scale=0.4]
	\begin{pgfonlayer}{nodelayer}
		\node [style=wn] (3) at (0, 0) {$#1$};
		\node [style=none] (4) at (0, 0.75) {};
		\node [style=none] (5) at (-0.5, -0.75) {};
		\node [style=none] (6) at (0.5, -0.75) {};
		\node [style=label] (7) at (0, -0.75) {$\ldots$};
	\end{pgfonlayer}
	\begin{pgfonlayer}{edgelayer}
		\draw (3) to (4.center);
		\draw (3) to (5.center);
		\draw (6.center) to (3);
	\end{pgfonlayer}
\end{tikzpicture}
}

\newcommand{\fanout}{\begin{tikzpicture}[scale=0.4]
	\begin{pgfonlayer}{nodelayer}
		\node [style=wn] (3) at (0, 0) {$x_i$};
		\node [style=none] (5) at (-0.5, 1) {};
		\node [style=none] (6) at (0.5, 1) {};
		\node [style=label] (7) at (0, 1.25) {$\ldots$};
	\end{pgfonlayer}
	\begin{pgfonlayer}{edgelayer}
		\draw (3) to (5.center);
		\draw (6.center) to (3);
	\end{pgfonlayer}
\end{tikzpicture}
}

\newcommand{\arithnum}{\begin{tikzpicture}[scale=0.4]
	\begin{pgfonlayer}{nodelayer}
		\node [style=wn] (0) at (0, 0) {$a$};
		\node [style=none] (1) at (0, 1) {};
	\end{pgfonlayer}
	\begin{pgfonlayer}{edgelayer}
		\draw (1.center) to (0);
	\end{pgfonlayer}
\end{tikzpicture}
}

\newcommand{\discard}[1][0.5]{\begin{tikzpicture}[scale=#1]
	\begin{pgfonlayer}{nodelayer}
		\node [style=none] (0) at (0, 2) {};
		\node [style=none] (1) at (-0.75, 1) {};
		\node [style=none] (2) at (0.75, 1) {};
		\node [style=none] (3) at (-0.5, 0.75) {};
		\node [style=none] (4) at (0.5, 0.75) {};
		\node [style=none] (5) at (-0.25, 0.5) {};
		\node [style=none] (6) at (0.25, 0.5) {};
		\node [style=none] (7) at (0, 1) {};
	\end{pgfonlayer}
	\begin{pgfonlayer}{edgelayer}
		\draw [style=thick line] (7.center) to (0.center);
		\draw [style=thick line] (1.center) to (2.center);
		\draw [style=thick line] (3.center) to (4.center);
		\draw [style=thick line] (5.center) to (6.center);
	\end{pgfonlayer}
\end{tikzpicture}
}

% general stuff

\newcommand{\eqq}[1]{~\overset{\left(#1\right)}{=}~}
\newcommand{\linesep}{10pt}
\newcommand{\neweqline}{\displaybreak[0] \\[\linesep]}

\newcommand{\ketz}{|0\rangle}
\newcommand{\keto}{|1\rangle}
\newcommand{\braz}{\langle0|}
\newcommand{\brao}{\langle1|}

\newcommand{\ketp}{|+\rangle}
\newcommand{\ketm}{|-\rangle}
\newcommand{\brap}{\langle+|}
\newcommand{\bram}{\langle-|}

\newcommand{\kpsi}{|\psi\rangle}
\newcommand{\bpsi}{\langle \psi |}


\newcommand{\CC}{\mathcal{C}}
\newcommand{\mone}{\mathbf{1}}


\newcommand{\Csq}{\mathbb{C}^2}
\newcommand{\polyring}[1][n]{\mathcal{P}_{#1}}
\newcommand{\csring}[1][n]{\tilde{S}_{#1}}
\newcommand{\pp}[1][]{p_{\psi_{#1}}}

\newcommand{\sharpp}{\#\mathsf{P}}

% Stuff I added
\usepackage{multicol, amsthm, amsfonts}


%\newtcbtheorem[auto counter, number within=section]{tcthm}{Theorem}%
%{colback=blue!5,colframe=blue!35!white,fonttitle=\bfseries}{thm}

%\newtcbtheorem[number within=section,
%                  use counter from=tcthm]{tclem}{Lemma}%
%{colback=blue!5,colframe=blue!35!white,fonttitle=\bfseries}{lem}

%\newtcbtheorem[number within=section,
%                  use counter from=tcthm]{tcprop}{Proposition}%
%{colback=blue!5,colframe=blue!35!white,fonttitle=\bfseries}{prop}


%\newtcbtheorem[number within=section,
%                  use counter from=tcthm]{tcdef}{Definition}%
%{colback=green!5,colframe=green!45!black,fonttitle=\bfseries}{def}


%\tcolorboxenvironment{proof}{
%  fonttitle=\bfseries,
%  breakable,
%  enhanced,
%  top=0mm,
%  boxrule=0pt,frame empty,
%  borderline west={1pt}{0pt}{blue!35!white},
%  coltitle=blue,
%  colback=blue!1,
%  rounded corners
%}

% copy for lemma and prop and def (but different colour) once done


\definecolor{zx_grey}{RGB}{211,211,211}
\definecolor{zx_pink}{RGB}{255,200,240}
\definecolor{zx_green}{RGB}{216,248,216}
\definecolor{zx_green2}{RGB}{180,248,180}
\definecolor{zx_red}{RGB}{232,165,165}


%\newcolumntype{g}{>{\columncolor{zx_green}}c}
%\newcolumntype{r}{>{\columncolor{zx_red}}c}

\newcommand*\Eval[2]{\left.#1\right\rvert_{#2}}

\newcommand{\lowerbox}[2][5]{\raisebox{-#1pt}{#2}}

\newcommand{\Hilb}{\operatorname{Hilb}}
\newcommand{\Control}{\textsf{Ctrl}}
