\section{Polynomials}

\subsection{Rings}


Let $\tilde{E_n}$ be the set of controlled square matrices on $n$ qubits. The goal of this section is to prove that the addition and multiplication operations introduced above induce a ring on $\tilde{E_n}$. Before doing so, we prove a few important lemmas. The first lemma enables us to copy controlled matrices. Note that all proofs can be found in the appendix.

\begin{lemma}\label{lem:csq_copy}
    For any square matrix $D$, 
    \begin{equation}\label{eq:csq_copy}
    \tikzfig{tikz/con/csq_copy}
\end{equation}
\end{lemma}



Now we show that controlled matrix addition and multiplication satisfy the ring axioms. Associativity of $+, \times$ follow immediately from (\ref{rule:Aso}, \ref{rule:S1}), respectively. Commutativity of addition follows from the commutativity of matrix addition.

\begin{lemma}\label{lem:csq_add_comm}
    Let $M_1, M_2$ be $n \times n$ matrices. 
    \begin{equation}\label{eq:csq_add_comm}
        \tikzfig{tikz/con/csq_add_comm_statement}
    \end{equation}
\end{lemma}


The additive identity is defined as $\redzeroup \otimes I_n$:
\begin{equation*}
    \tikzfig{tikz/con/csq_add_id}
\end{equation*}


The multiplicative identity is defined very similarly as $\greenzeroup \otimes I_n$. The existence of additive inverses relies on the copying lemma from before.

\begin{lemma}
    The additive inverse of $\tilde{M}$ is $\raisebox{-5pt}{\numbergate[-1]} \circ \tilde{M} $
\end{lemma}
\begin{proof}
    \begin{equation*}
        \tikzfig{tikz/con/csq_add_inv}
    \end{equation*}
\end{proof}

Finally, we prove distributivity.

\begin{lemma}\label{lem:cm_dist}
    \begin{equation*}
        \tikzfig{tikz/con/csq_dist_statement}
    \end{equation*}
\end{lemma}


Combining the lemmas of this section shows that controlled matrices form a ring. A similar result can be shown for controlled states. Once again, we start with the ability to copy controlled states. 
\begin{lemma}\label{lem:cs_copy}
    For any state $\psi$,
    \begin{equation}\label{eq:cs_copy}
        \tikzfig{tikz/con/cs_copy}
    \end{equation}
\end{lemma}

Many of the ring axioms follow directly from basic ZXW rules. For example we can show commutativity of addition as follows:

\begin{lemma}\label{lem:cs_add_comm}
    For $n$-partite states $\psi_1, \psi_2$, $\tilde{\psi_1} \boxplus \tilde{\psi_2} = \tilde{\psi_2} \boxplus \tilde{\psi_1}  $
\end{lemma}

Associativity of $\boxplus$ follows similarly, using (\ref{rule:Aso}). Next we have the additive identity.

\begin{lemma}\label{lem:cs_add_id}
    $\tilde{\psi} \boxplus\mathbf{\tilde{0}} = \tilde{\psi}$
\end{lemma}


The additive inverse is defined similarly to the case of controlled matrices. 


\begin{lemma}\label{lem:cs_add_inv}
    For a controlled state $\tilde{\psi}$, its additive inverse is $\tilde{\psi} \circ \raisebox{-5pt}{\numbergate[-1]}$
\end{lemma}


Associativity and commutativity of $\boxtimes$ follow as before, using (\ref{rule:S1}) for $\lowerbox{\zspid}$. Finally, we must prove distributivity.


\begin{lemma}\label{lem:cs_dist}
    $\tilde{\psi_1} \boxtimes (\tilde{\psi_2} \boxplus \tilde{\psi_3}) = (\tilde{\psi_1} \boxtimes \tilde{\psi_2}) \boxplus (\tilde{\psi_1} \boxtimes \tilde{\psi_3})$
\end{lemma}




\subsection{Arithmetic}

Its been known since 2011 that $\lowerbox{\wspid}, \lowerbox{\zspid}$ can be used to add and multiply numberstates $\numberstate$, respectively \cite{coecke2011ghz}. In the previous section we saw that $\lowerbox{\wspid}, \lowerbox{\zspid}$ can moreover be used to copy controlled diagrams. In this section, we explain this connection by demonstrating that controlled states are in fact isomorphic to multilinear polynomials. Firstly, we describe how to interpret certain ZXW diagrams as polynomials. Consider the following diagrams:

\begin{equation*}
    \tikzfig{tikz/poly/eg1}
\end{equation*}

If we treat the bottom wires as an indeterminate $x$, we can read these bottom-up as computing $x - 1$ and $2x + 3$, respectively. Moreover, since these diagrams are both controlled states, they can be added together,  yield a diagram resembling $3x + 2$:
\begin{equation*}
    \tikzfig{tikz/poly/eg4}
\end{equation*}

When trying to multiply these diagrams, rather than getting $(x-1)(2x+3) = 2x^2 + x - 3$, we instead get $x - 3$.
\begin{equation*}
    \tikzfig{tikz/poly/eg5}
\end{equation*}

The reason for the missing $2x^2$ term is that (\ref{eq:kill_quad}) implies $x^2 = 0$. Other than that, controlled state arithmetic appears to faithfully reflect polynomial arithmetic. To help formalise this correspondence, we introduce the following definition.

\begin{definition}
    A ZXW diagram with a single input on top is \textbf{arithmetic} if it contains only  $\lowerbox{\idwire}$, $\lowerbox{\swap}$ wires, $\lowerbox[7]{\wspids}$, $\lowerbox[7]{\zspids}$, $\lowerbox{\coWs}$ nodes and $\numberstate$ boxes.
\end{definition}


To interpret an arithmetic ZXW diagram as an arithmetic expression, read $\wspid$ as $+$, $\zspid$ as $\times$, $\numberstate$ as the number $a$, $\coW$ as fanout and output/bottom wires as variables $x_1, ..., x_n$ numbered from left to right. The following lemma establishes that all arithmetic diagrams are controlled states:
\begin{lemma}
    For any arithmetic diagram $A$, \begin{equation}\label{eq:arith_cs}
        \tikzfig{tikz/poly/arith0_statement}
    \end{equation}
\end{lemma}
\begin{proof}
    By definition, other than wires $A$ contains only $\raisebox{-7pt}{\wspids}$, $\raisebox{-7pt}{\zspids}$, $\raisebox{-5pt}{\coWs}$, and $\raisebox{-3pt}{\numberstate}$. All $\raisebox{-3pt}{\numberstate}$'s can be removed with (\ref{rule:Ept}). Meanwhile all the spiders copy $\raisebox{-5pt}{\redzero}$ due to (\ref{rule:Bs0}, \ref{rule:K0}, \ref{eq:wid}) respectively.
\end{proof}

Just as it is typical to represent a polynomial in normal form as a sum of products, it is possible to rewrite every arithmetic diagram into a normal form as a single $\lowerbox{\wspids}$, followed by a layer of $\lowerbox{\zspids}$, followed by a layer of $\numberstate, \lowerbox{\coWs}$. 

\begin{definition}
    An $n$-output arithmetic diagram is said to be written in \textbf{polynormal form} (PNF) if it looks like:
    \begin{equation*}
        \tikzfig{tikz/poly/pnf}
    \end{equation*}

    The $i$th coefficient $a_i$ is connected to the $k$th $\lowerbox{\coWs}$ iff the $k$th bit in the binary expansion of $i$ is 1. 
\end{definition}

This normal form is very closely related to the completeness normal form (see \cite{poor2023completeness}). Simply applying (\ref{rule:TA}) to the $\lowerbox{\coW}$s at the bottom of a PNF and fusing the number boxes gives a CoNF diagram. The reason we introduce the definition of a PNF is that it is an arithmetic diagram and therefore has a more immediate arithmetic interpretation. The reason for the specific connectivity condition is that it enables a PNF to directly represent its own matrix.


\begin{prop}\label{prop:vec_pnf}
    \begin{gather}\label{eq:pnf_vec}
        \tikzfig{tikz/poly/pnf} ~=~ 
        \begin{bmatrix}
            1 &  a_0 \\ 0 & a_1 \\ ... & ... \\ 0 & a_{2^n-1}
        \end{bmatrix}
    \end{gather}
\end{prop} 

\begin{proof}
    See appendix \ref*{sec:appiso}
\end{proof}

Thus, every controlled state can be represented as at least one arithmetic diagram (namely, its PNF). Moreover, we now show that any other arithmetic diagram can always be rewritten to its PNF.


\begin{prop}\label{prop:uni_pnf}
    All arithmetic diagrams can be written into PNF
\end{prop}

\begin{proof}
    See appendix \ref*{sec:appiso}
\end{proof}

\subsection{Isomorphism}

At last we can prove the isomorphism. Throughout we shall let $\polyring$ denote the ring $\mathbb{C}[x_1, ..., x_{n}]/(x_1^2, ..., x_{n}^2)$.


\begin{thm}\label{thm:iso}
    There is an isomorphism $\polyring \simeq \tilde{S_n}$
\end{thm}

First, we shall define the map $\phi_n: \polyring \to \csring$ before proving it induces an isomorphism. $\phi_n$ is defined to map an arbitrary polynomial $p(x_1, ..., x_n) = a_0 + a_1x_n + ... + a_{2^n-1}x_1x_2...x_n$ to the following PNF:
    \begin{equation*}
        \phi_n(p) ~=~ \tikzfig{tikz/poly/pnf}
    \end{equation*}

Some important special cases are mapping scalars $a \in \mathbb{C}$:
 \begin{equation*}
        \tikzfig{tikz/poly/homproof/homdef1}
\end{equation*}

And mapping indeterminates $x_i$:
    \begin{equation*}
        \tikzfig{tikz/poly/homproof/homdef2}
    \end{equation*}

The full proof is found in appendix \ref*{sec:appiso}.
