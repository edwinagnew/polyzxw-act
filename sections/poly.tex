\section{Polynomials}

\subsection{Rings}

\textbf{Most of this will be moved to the appendix}

Let $\tilde{E_n}$ be the set of controlled square matrices on $n$ qubits. The goal of this section is to prove that the addition and multiplication operations introduced above induce a ring on $\tilde{E_n}$. Before doing so, we prove a few important lemmas. The first lemma enables us to copy controlled matrices. 

\begin{lemma}
    For any square matrix $D$, 
    \begin{equation}\label{eq:cm_copy}
    \tikzfig{tikz/con/c_dcopy}
\end{equation}
\end{lemma}

\begin{proof}
    First of all, using (\ref{rule:BZW}) we can rewrite the LHS to
    \begin{equation*}
        \tikzfig{tikz/con/c_dcopy2}
    \end{equation*}

    Then clearly 
    \begin{equation*}
        \tikzfig{tikz/con/c_dcopy3}
    \end{equation*}

    Meanwhile, 
    \begin{equation*}
        \tikzfig{tikz/con/c_dcopy4}
    \end{equation*}

    Thus the two sides are equal over the Z basis and so are equal as diagrams.
\end{proof}

Now we show that controlled matrix addition and multiplication satisfy the ring axioms. Associativity of $+, \times$ follow immediately from (\ref{rule:Aso}, \ref{rule:S1}), respectively. Commutativity of addition follows from the commutativity of matrix addition.

\begin{lemma}
    Let $M_1, M_2$ be $n \times n$ matrices. 
    \begin{equation}\label{eq:cm_add_comm}
        \tikzfig{tikz/con/cm_add_comm_statement}
    \end{equation}
\end{lemma}
\begin{proof}
     We prove by plugging red and commutativity of matrix addition. By definition of controlled matrices, plugging $\lowerbox{\redzero}$ gives $I_n$ on both sides. Meanwhile, plugging $\lowerbox{\redpi}$ gives:
     \begin{equation*}
        \tikzfig{tikz/con/cm_add_comm}
    \end{equation*}
\end{proof}

The additive identity is defined as $\redzeroup \otimes I_n$:
\begin{equation*}
    \tikzfig{tikz/con/cm_add_id}
\end{equation*}


The multiplicative identity is defined very similarly as $\greenzeroup \otimes I_n$. The existence of additive inverses relies on the copying lemma from before.

\begin{lemma}
    The additive inverse of $\tilde{M}$ is $\raisebox{-5pt}{\numbergate[-1]} \circ \tilde{M} $
\end{lemma}
\begin{proof}
    \begin{equation*}
        \tikzfig{tikz/con/cm_add_inv}
    \end{equation*}
\end{proof}

Finally, we prove distributivity.

\begin{lemma}
    \begin{equation*}
        \tikzfig{tikz/con/cm_dist_statement}
    \end{equation*}
\end{lemma}
\begin{proof}
    \begin{equation*}
        \tikzfig{tikz/con/cm_dist}
    \end{equation*}
\end{proof}


Combining the lemmas of this section shows that controlled matrices form a ring. A similar result can be shown for controlled states. Once again, we start with the ability to copy controlled states. 
\begin{lemma}
    For any state $\psi$,
    \begin{equation}\label{eq:cs_copy}
        \tikzfig{tikz/con/cs_copy}
    \end{equation}
\end{lemma}
\begin{proof}
    As before, plugging $|0\rangle$ gives
    \begin{equation*}
        \tikzfig{tikz/con/cs_copy1}
    \end{equation*}

    Meanwhile, plugging $|1\rangle$ gives
        \begin{equation*}
        \tikzfig{tikz/con/cs_copy2}
    \end{equation*}

    Completing the proof
\end{proof}

Many of the ring axioms follow directly from basic ZXW rules. For example we can show commutativity of addition as follows:

\begin{lemma}
    For $n$-partite states $\psi_1, \psi_2$, $\tilde{\psi_1} \boxplus \tilde{\psi_2} = \tilde{\psi_2} \boxplus \tilde{\psi_1}  $
\end{lemma}

\begin{proof}
    \tikzfig{tikz/con/cs_add_comm}
\end{proof}

Associativity of $\boxplus$ follows similarly, using (\ref{rule:Aso}). Next we have the additive identity.

\begin{lemma}
    $\tilde{\psi} \boxplus\mathbf{\tilde{0}} = \tilde{\psi}$
\end{lemma}
\begin{proof}
    It is clear that $\lowerbox[10]{\zeroproj}$ is the controlled state $\tilde{\mathbf{0}}$. 
    
    Then we have:
    \begin{equation*}
        \tikzfig{tikz/con/cs_add_id}
    \end{equation*}

\end{proof}

The additive inverse is defined similarly to the case of controlled matrices. 


\begin{lemma}
    For a controlled state $\tilde{\psi}$, its additive inverse is $\tilde{\psi} \circ \raisebox{-5pt}{\numbergate[-1]}$
\end{lemma}
\begin{proof}
    $\tilde{\psi} \circ \raisebox{-5pt}{\numbergate[-1]}$ is still a controlled state since $\raisebox{-5pt}{\numbergate[-1]}$ does nothing to $\raisebox{-5pt}{\redzero}$. Then $\tilde{\psi} \circ \raisebox{-5pt}{\numbergate[-1]}$ inverts $\tilde{\psi}$ since:
    \begin{equation*}
        \tikzfig{tikz/con/cs_add_inv}
    \end{equation*}
\end{proof}

Associativity and commutativity of $\boxtimes$ follow as before, using (\ref{rule:S1}) for $\lowerbox{\zspid}$. Finally, we must prove distributivity.


\begin{lemma}
    $\tilde{\psi_1} \boxtimes (\tilde{\psi_2} \boxplus \tilde{\psi_3}) = (\tilde{\psi_1} \boxtimes \tilde{\psi_2}) \boxplus (\tilde{\psi_1} \boxtimes \tilde{\psi_3})$
\end{lemma}


\begin{proof}


    \begin{gather*} %\label{eq:cs_dist}
        \tilde{\psi_1} \boxtimes (\tilde{\psi_2} \boxplus \tilde{\psi_3}) ~=~ \tikzfig{tikz/distproof/d1} ~
        \eqq{\ref{rule:BZW}} ~ \tikzfig{tikz/distproof/d2} \neweqline ~=~ \tikzfig{tikz/distproof/d3} ~
        \eqq{\ref{eq:cs_copy}} ~ \tikzfig{tikz/distproof/d4} \neweqline ~=~ \tikzfig{tikz/distproof/d5}
        ~=~ \tikzfig{tikz/distproof/d6} \neweqline ~=~ (\tilde{\psi_1} \boxtimes \tilde{\psi_2}) \boxplus (\tilde{\psi_1} \boxtimes \tilde{\psi_3})
    \end{gather*} 

\end{proof}


\subsection{Arithmetic}

Recall from (\ref{rule:ADD}, \ref{rule:S1}) that $\lowerbox{\wspid}, \lowerbox{\zspid}$ can be used to add and multiply numberstates $\numberstate$, respectively. In the previous section we saw that $\lowerbox{\wspid}, \lowerbox{\zspid}$ can moreover be used to copy controlled diagrams. In this section, we explain this connection by demonstrating that controlled states are in fact isomorphic to multilinear polynomials. Firstly, we describe how to interpret certain ZXW diagrams as polynomials. Consider the following diagrams:

\begin{equation*}
    \tikzfig{tikz/poly/eg1}
\end{equation*}

If we treat the bottom wires as an indeterminate $x$, we can read these bottom-up as computing $x - 1$ and $2x + 3$, respectively. Moreover, since these diagrams are both controlled states, they can be added together,  yield a diagram resembling $3x + 2$:
\begin{equation*}
    \tikzfig{tikz/poly/eg4}
\end{equation*}

When trying to multiply these diagrams, rather than getting $(x-1)(2x+3) = 2x^2 + x - 3$, we instead get $x - 3$.
\begin{equation*}
    \tikzfig{tikz/poly/eg5}
\end{equation*}

The reason for the missing $2x^2$ term is that (\ref{eq:kill_quad}) implies $x^2 = 0$. Other than that, controlled state arithmetic appears to faithfully reflect polynomial arithmetic. To help formalise this correspondence, we introduce the following definition.

\begin{definition}
    A ZXW diagram with a single input on top is \textbf{arithmetic} if it contains only  $\lowerbox{\idwire}$, $\lowerbox{\swap}$ wires, $\lowerbox[7]{\wspids}$, $\lowerbox[7]{\zspids}$, $\lowerbox{\coWs}$ nodes and $\numberstate$ boxes.
\end{definition}


To interpret an arithmetic ZXW diagram as an arithmetic expression, read $\wspid$ as $+$, $\zspid$ as $\times$, $\numberstate$ as the number $a$, $\coW$ as fanout and output/bottom wires as variables $x_1, ..., x_n$ numbered from left to right. The following lemma establishes that all arithmetic diagrams are controlled states:
\begin{lemma}
    For any arithmetic diagram $A$, \begin{equation*}
        \tikzfig{tikz/poly/arith0_statement}
    \end{equation*}
\end{lemma}
\begin{proof}
    By definition, other than wires $A$ contains only $\raisebox{-7pt}{\wspids}$, $\raisebox{-7pt}{\zspids}$, $\raisebox{-5pt}{\coWs}$, and $\raisebox{-3pt}{\numberstate}$. All $\raisebox{-3pt}{\numberstate}$'s can be removed with (\ref{rule:Ept}). Meanwhile all the spiders copy $\raisebox{-5pt}{\redzero}$ due to (\ref{rule:Bs0}, \ref{rule:K0}, \ref{eq:wid}) respectively.
\end{proof}

Just as it is typical to represent a polynomial in normal form as a sum of products, it is possible to rewrite every arithmetic diagram into a normal form as a single $\lowerbox{\wspids}$, followed by a layer of $\lowerbox{\zspids}$, followed by a layer of $\numberstate, \lowerbox{\coWs}$. 

\begin{definition}
    An $n$-output arithmetic diagram is said to be written in \textbf{polynormal form} (PNF) if it looks like:
    \begin{equation*}
        \tikzfig{tikz/poly/pnf}
    \end{equation*}

    The $i$th coefficient $a_i$ is connected to the $k$th $\lowerbox{\coWs}$ iff the $k$th bit in the binary expansion of $i$ is 1. 
\end{definition}

This normal form is very closely related to the completeness normal form (CoNF, see Definition \ref{def:cnf}). Simply applying (\ref{rule:TA}) to the $\lowerbox{\coW}$s at the bottom of a PNF and fusing the number boxes gives a CoNF diagram. The reason we introduce the definition of a PNF is that it is an arithmetic diagram and therefore has a more immediate arithmetic interpretation. The reason for the specific connectivity condition is that it enables a PNF to directly represent its own matrix.


\begin{prop}
    \begin{gather}\label{eq:pnf_vec}
        \tikzfig{tikz/poly/pnf} ~=~ 
        \begin{bmatrix}
            1 &  a_0 \\ 0 & a_1 \\ ... & ... \\ 0 & a_{2^n-1}
        \end{bmatrix}
    \end{gather}
\end{prop} 

\begin{proof}
    We prove by induction on $n$.
    
    For the base case, $n=0$. The only PNF with no outputs is a number so we have: $$\numberstate[a_0] = \begin{bmatrix}
        1 & a_0
    \end{bmatrix}$$ as desired.
    
    For inductive hypothesis, we assume that (\ref{eq:pnf_vec}) holds for every PNF on $n$ outputs. We use this hypothesis to extend it to PNFs with $n+1$ outputs. 
    
    Let $D$ be an arbitrary PNF with $n+1$ outputs. Firstly, observe that $x_{n+1}$ is connected to only the odd coefficients $\{a_{2k+1}\}$ since these are exactly the indices with $1$ in the least significant bit. Thus we can rewrite:
    \begin{gather*}
        \raisebox{-20pt}{\ddiag} ~=~ \tikzfig{tikz/poly/lemmas/uni10} \neweqline ~=~ \tikzfig{tikz/poly/lemmas/uni11} \neweqline ~\eqq{\ref{rule:BZW}}~ \tikzfig{tikz/poly/lemmas/uni12} \neweqline 
        ~=~ \tikzfig{tikz/poly/lemmas/uni13}
    \end{gather*}
    Where $D_{even}, D_{odd}$ are PNF diagrams. Since they are over $n$ variables, we can apply the inductive hypothesis and obtain:
    \begin{equation}\tag{*}
        D_{even} = \begin{bmatrix}
            1 & a_0 \\ 0 & a_2 \\ ... & ... \\ 0 & a_{2^{n+1}-2}
        \end{bmatrix}, 
        D_{odd} = \begin{bmatrix}
            1 & a_1 \\ 0 & a_3 \\ ... & ... \\ 0 & a_{2^{n+1}-1}
        \end{bmatrix}
    \end{equation}


    Next, plugging red we observe:
    \begin{equation*}
        \tikzfig{tikz/poly/lemmas/uni2}
    \end{equation*}
    Meanwhile,
    \begin{equation*}
        \tikzfig{tikz/poly/lemmas/uni3}
    \end{equation*}

    Summing these together,
    \begin{gather*}
        \tikzfig{tikz/poly/lemmas/uni4}
        \neweqline ~=~ (D_{even} \otimes \ketz) + (D_{odd}\keto \brao \otimes  \keto) \neweqline \eqq{*}~ \begin{bmatrix}
            1 & a_0 \\ 0 & a_2 \\ ... & ... \\ 0 & a_{2^{n+1}-2}
        \end{bmatrix} \otimes \ketz + \begin{bmatrix}
            0 & a_1 \\ 0 & a_3 \\ ... & ... \\ 0 & a_{2^{n+1}-1}
        \end{bmatrix} \otimes \keto  \neweqline
        ~=~ \begin{bmatrix}
            1 & a_0 \\ 0 & 0 \\ 0 & a_2 \\ 0 & 0 \\ ... & ... \\ 0 & a_{2^{n+1}-2} \\ 0 & 0
        \end{bmatrix} + \begin{bmatrix}
            0 & 0 \\ 0 & a_1 \\ 0 & 0 \\ 0 & a_3 \\ ... & ... \\ 0 & 0 \\ 0 & a_{2^{n+1}-1}
        \end{bmatrix} 
        ~=~ \begin{bmatrix}
            1 & a_0 \\ 0 & a_1 \\ 0 & a_2 \\ 0 & a_3 \\ ... & ... \\ 0 & a_{2^{n+1}-2} \\ 0 & a_{2^{n+1} -1}
        \end{bmatrix}
    \end{gather*}

    Completing the inductive step.
\end{proof}

Thus, every controlled state can be represented as at least one arithmetic diagram (namely, its PNF). Moreover, we now show that any other arithmetic diagram can always be rewritten to its PNF.


\begin{prop}
    All arithmetic diagrams can be written into PNF
\end{prop}

\begin{proof}
    Let $A$ be an arithmetic diagram. If $A = \numberstate$, we are done. 
    
    Otherwise, $A$ has at least one output. First, we shall rewrite $A$ into three layers, consisting of: (1) a single W at the top, (2) a layer of $\raisebox{-5pt}{\zspids}$ and (3) a layer of $\numberstate$'s and $\raisebox{-5pt}{\coWs}$'s. Then we shall collect terms and order the boxes to produce a PNF. 

    If the top of $A$ is not already $\raisebox{-5pt}{\wspids}$, it must be $\raisebox{-5pt}{\zspids}$. It cannot be $\numberstate$ since the remaining arithmetic diagram would then have no inputs which is impossible. It cannot be $\raisebox{-5pt}{\coWs}$ since there is only one input and arithmetic diagrams cannot contain $\ccap$. Thus we can rewrite:
    \begin{enumerate}[label={(\arabic*)}]
    \item $\tikzfig{tikz/poly/lemmas/algtop1}$
    \end{enumerate}

    (1) guarantees there is a W at the top. We shall now repeatedly apply rewrites underneath the W until there are exactly three layers. Assume that fusion is applied as much as possible between each stage and (\ref{eq:kill_quad}) is applied and simplified with (\ref{rule:K0}) to remove $\raisebox{-8pt}{\xsq}$ whenever possible. Then for as long as there are at least 4 layers, we can apply one of the following rewrites:
        \begin{enumerate}[resume, label={(\arabic*)}]
            \item $\tikzfig{tikz/poly/lemmas/algcases1}$
            \item $\tikzfig{tikz/poly/lemmas/algcases2}$
            \item $\tikzfig{tikz/poly/lemmas/algcases3}$
            \item $\tikzfig{tikz/poly/lemmas/algcases4}$
            \item $\tikzfig{tikz/poly/lemmas/algcases5}$
        \end{enumerate}

    \medskip
    
    Clearly, we can only stop applying these rules once $A$ is a sum of products of copies. Steps (2) and (3) ensure the top of $A$ has such a structure and steps (4) - (6) ensure that there is nothing beneath the $\lowerbox{\coWs}$'s . To see that this will always terminate, observe that (2) and (3) preserve the depth of $A$ while (4), (5), (6) all decrease it. (2) and (3) can only be applied a finite number of times before another simplification must be used. So repeatedly applying these rewrites must eventually shrink the depth down to $3$, as desired. Finally, to put $A$ in PNF we must:
    \begin{enumerate}[resume, label={(\arabic*)}]
        \item Collect terms: whenever there are two boxes connected to exactly the same set of $\raisebox{-5pt}{\coWs}$'s, use (\ref{eq:cpk_add}) to fuse them together. 
        \item Pad: use (\ref{eq:zerobox}) to insert $\raisebox{-5pt}{\numbergate[0]}$ for any connectivities that do not exist in $A$.
        \item Reorder: use (\ref{rule:Sym}) to reorder coefficients into the canonical order.
    \end{enumerate}

    Step (7) ensures that every $\raisebox{-5pt}{\zspids}$ has unique connectivity. Step (8) ensures there are exactly $2^n$ coefficients so that step (9) can order them in the appropriate way. 

    Thus $A$ has been written in PNF, completing the proof.
    

\end{proof}

\subsection{Isomorphism}

At last we can prove the isomorphism. Throughout we shall let $\polyring$ denote the ring $\mathbb{C}[x_1, ..., x_{n}]/(x_1^2, ..., x_{n}^2)$.


\begin{thm}
    There is an isomorphism $\polyring \simeq \tilde{S_n}$
\end{thm}

First, we shall define the map $\phi_n: \polyring \to \csring$ before proving it induces an isomorphism. $\phi_n$ is defined to map an arbitrary polynomial $p(x_1, ..., x_n) = a_0 + a_1x_n + ... + a_{2^n-1}x_1x_2...x_n$ to the following PNF:
    \begin{equation*}
        \phi_n(p) ~=~ \tikzfig{tikz/poly/pnf}
    \end{equation*}

Some important special cases are mapping scalars $a \in \mathbb{C}$:
 \begin{equation*}
        \tikzfig{tikz/poly/homproof/homdef1}
\end{equation*}

And mapping indeterminates $x_i$:
    \begin{equation*}
        \tikzfig{tikz/poly/homproof/homdef2}
    \end{equation*}

We can now prove this yields an isomorphism. 

\begin{proof}
    
    First, we show $\phi_n$ is a homomorphism, i.e. \begin{multline*}
        \forall p, q \in \polyring, \phi_n(p + q) = \phi_n(p) \boxplus \phi_n(q) ,\quad \phi_n(p \times q) = \phi_n(p) \boxtimes \phi_n(q)
    \end{multline*} The strategy for the proof will be an induction on $n$. 

    \medskip
    
    \textbf{Base case:}
    We have not defined controlled states for $n=0$, so the base case begins with $n=1$.
    Let $p, q \in \polyring[1]$. Write as $p(x_1) = a_0 + a_1x_1, q(x_1) = b_0 + b_1x_1$, where $a_0, a_1, b_0, b_1 \in \mathbb{C}$. Then since $p + q = a_0 + b_0 + (a_1 + b_1)x_1$,
    \begin{equation*}
        \tikzfig{tikz/poly/homproof/hombaseadd}
    \end{equation*}

    Meanwhile, since $p \times q = a_0a_1 + (a_0b_1 + a_1b_0)x_1$, 
    \begin{gather*}
        \phi_1(p) \boxtimes \phi_1(q) ~=~ \tikzfig{tikz/poly/homproof/bt1} ~\eqq{\ref{eq:dbl_dist}}~ \tikzfig{tikz/poly/homproof/bt2} \neweqline 
        ~\eqq{\ref{eq:cd_copy}}~ \tikzfig{tikz/poly/homproof/bt3} ~\eqq{\ref{rule:Pcy}}~ \tikzfig{tikz/poly/homproof/bt4} \neweqline
        ~\eqq{\ref{eq:kill_quad}}~ \tikzfig{tikz/poly/homproof/bt5} ~\eqq{\ref{eq:0times}}~ \tikzfig{tikz/poly/homproof/bt6} ~\eqq{\ref{eq:cp_add}}~ \tikzfig{tikz/poly/homproof/bt7} \neweqline ~= \phi_1(p \times q)
    \end{gather*}


    Completing the base case.

    \medskip


    \textbf{Inductive step:}

    Let $Hom(n)$ assert than $\phi_n$ is a homomorphism.  Then for the inductive step we wish to prove that $\forall n, Hom(n) \implies Hom(n+1)$.

    The proof relies on the recursive definition of $R[x_1, x_2] = R[x_1][x_2]$, for any ring $R$, to rewrite an arbitrary polynomial $p(x_1, ..., x_{n+1}) = a_0 + a_1x_{n+1} + ... + a_{2^{n+1}-1}x_1x_2...x_{n+1} \in \polyring[n+1]$ as $p(x_{n+1}) = p_0 + p_1x_{n+1}$, where $p_0, p_1 \in \polyring$. This allows the $p_i$ to be treated similarly to the scalars in the base case. To emphasise this, they will be drawn in green boxes. To help distinguish when an operation is covered by the inductive hypothesis, the wires for variables $x_1, ..., x_n$ will be drawn in light blue, while the $x_{n+1}$ wires will be drawn in black. Thus the inductive hypothesis states that:
    \begin{equation}\label{eq:ih1}\tag{IH1}
        \tikzfig{tikz/poly/homproof/ih1}
    \end{equation}
    \begin{equation}\label{eq:ih2}\tag{IH2}
        \tikzfig{tikz/poly/homproof/ih2}
    \end{equation}

    Let $p(x_{n+1}) = p_0 + p_1x_{n+1}, q(x_{n+1}) = q_0 + q_1x_{n+1}$, where $p_0, p_1, q_0, q_1 \in \polyring$. Then for addition:
    \begin{gather*}
        \phi_{n+1}(p) \boxplus \phi_{n+1}(q) ~=~ \tikzfig{tikz/poly/homproof/sa1} ~\eqq{\ref{rule:Aso}}~ \tikzfig{tikz/poly/homproof/sa2} \neweqline
        ~ \eqq{IH1} ~ \tikzfig{tikz/poly/homproof/sa3} ~\eqq{\ref{rule:BZW}}~ \tikzfig{tikz/poly/homproof/sa4} ~\eqq{IH1}~ \tikzfig{tikz/poly/homproof/sa5} \neweqline
        ~=~ \phi_{n+1}(p_0 + q_0 + (p_1 + q_1)x_{n+1}) ~=~ \phi_{n+1}(p + q)
    \end{gather*}


    Similarly, for multiplication:
    \begin{gather*}
    \phi_{n+1}(p) \boxtimes \phi_{n+1}(q) ~=~ \tikzfig{tikz/poly/homproof/st1} \eqq{\ref{eq:dbl_dist}}  ~ \tikzfig{tikz/poly/homproof/st2} \neweqline
    \eqq{\ref{eq:cs_copy}} ~ \tikzfig{tikz/poly/homproof/st3} \eqq{\ref{eq:ih2}} ~ \tikzfig{tikz/poly/homproof/st4} \neweqline
    \eqq{\ref{rule:BZW}} ~ \tikzfig{tikz/poly/homproof/st5} ~ = ~ \tikzfig{tikz/poly/homproof/st6} \neweqline
     \eqq{\ref{eq:cs_copy}} ~ \tikzfig{tikz/poly/homproof/st7} ~ \eqq{\ref{eq:ih2}} ~ \tikzfig{tikz/poly/homproof/st8} \neweqline
    \eqq{\ref{rule:BZW}} ~ \tikzfig{tikz/poly/homproof/st9} ~ = ~ \tikzfig{tikz/poly/homproof/st10} \neweqline
    \eqq{\ref{eq:kill_quad}} ~ \tikzfig{tikz/poly/homproof/st11} ~ \eqq{\ref{lem:arith_cs}, \ref{eq:wid}} ~ \tikzfig{tikz/poly/homproof/st12} \neweqline ~ \eqq{\ref{eq:ih2}} ~ \tikzfig{tikz/poly/homproof/st13}
      ~ \eqq{\ref{rule:BZW}} ~ \tikzfig{tikz/poly/homproof/st14} ~ \eqq{\ref{eq:ih1}} ~ \tikzfig{tikz/poly/homproof/st15} \neweqline
     ~=~ \phi_{n+1}(p_0q_0 + (p_0q_1 + p_1q_0)x_{n+1} )
     ~=~ \phi_{n+1}(p \times q)
    \end{gather*}

    \medskip

    This completes the inductive step, proving that $\forall n > 1$, $\phi_n$ is a homomorphism.

    \bigskip

    Finally, to see $\phi_n$ is an isomorphism, we use proposition \ref{prop:uni_pnf} to write an arbitrary controlled state in PNF:
    \begin{gather*}
        \begin{bmatrix}
            1 & a_0 \\ 0 & a_1 \\ ... & .. \\ 0 & a_{2^{n}-1}
        \end{bmatrix}
        = \tikzfig{tikz/poly/pnf}
    \end{gather*}

    Then all we have to do is interpret it as the image of a polynomial:
    \begin{gather*}
        \tikzfig{tikz/poly/pnf} ~=~ \tikzfig{tikz/poly/homproof/iso4} \neweqline ~=~ \phi_{n}(a_0) + \phi_{n}(a_1x_{n}) + ... + \phi_{n}(a_{2^{n}-1x_1x_2...x_{n}}) \neweqline ~=~ \phi_{n}(a_0 + a_1x_{n} + ... + a_{2^{n}-1}x_1x_2...x_{n}) 
    \end{gather*}
\end{proof}