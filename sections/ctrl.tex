\section{Completeness for Factoring Controlled Operators}
Instead of the indeterminates being complex numbers represented by $\numberstate$'s, we can let them be same-size matrices represented by controlled square matrix diagrams.
We then have that:
\begin{thm}
    ZXW diagrams where the outputs of an arithmetic ZXW diagram are each plugged into controls of same-size controlled matrices, are isomorphic to multivariate polynomials over same-size square matrices with complex number coefficients.
    The rules for their completeness are the same subset of ZXW rules used for completeness for arithmetic diagrams in the ZXW-calculus in \Cref{thm:uni_pnf}, plus the controlled square matrix as a generator along with the four rewrite rules for it in \Cref{def:ctrlsqmat}, \Cref{lem:csq_copy}, and \Cref{lem:csq_add_comm}.
\end{thm}
\begin{proof}
    The proof is by the same algorithm for rewriting to PNF as \Cref{thm:uni_pnf}, modifying step (6) to copy controlled square matrices using \Cref{lem:csq_copy}, using \Cref{lem:csq_add_comm} to commute controlled square matrices whose controls act on mutually exclusive sectors.
\end{proof}

As an application, we leverage both our rewrite rules for arithmetic ZXW diagrams, and for controlled diagrams, to \emph{factor} them.  For example, for same size square matrices $I, A, B$ and $a, b, c \in \mathbb{C}$:
\begin{gather*}
    \tikzfig{tikz/poly/matspoly}
\end{gather*}
Factoring Hamiltonians is important to optimise quantum algorithms for chemistry and physics simulations. However, previous graphical rewrites for factoring Hamiltonians had only been doable for Hamiltonians with concretely-specified matrix terms~\cite{shaikh2022sum}. This completeness result guarantees that for any Hamiltonian, even if its matrix terms are black-box, these graphical rewrite rules are capable of deriving any of its possible factorisations.