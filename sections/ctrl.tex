\section{Completeness for Factoring Controlled Operators}
Instead of the indeterminates being complex numbers represented by $\lowerbox{\numberstate}$'s, we can let them be same-size matrices represented by controlled square matrix diagrams.
We then have that:
\begin{thm}\label{thm:ctrl_pnf}
    ZXW diagrams where the outputs of an arithmetic ZW diagram are each plugged into controls of same-size controlled matrices, are isomorphic to multivariate polynomials over same-size square matrices with complex number coefficients.
    The rules for their completeness are the same subset of ZXW rules used for completeness for arithmetic diagrams in the ZXW-calculus in \Cref{thm:uni_pnf}, plus the controlled square matrix as a generator along with the four rewrite rules for it in \Cref{def:ctrlsqmat}, \Cref{lem:csq_copy}, and \Cref{lem:csqaddcomm}. (See \Cref{fig:zxw_rules}.)
\end{thm}
\begin{proof}
    The proof is by deriving a unique normal form, having fixed an (arbitrary) order on the same-size square matrix variables.
    The algorithm to arrive at this normal form starts by rewriting the diagram sans the controlled square matrices to PNF, by the procedure of \Cref{thm:uni_pnf}.
    Next, remove all $\lowerbox{\coWs}$'s by using \eqref{rule:CMcpy} to make copies of the controlled square matrices.
    Finally, use \eqref{rule:CMcom} to commute controlled square matrices whose controls act on mutually exclusive sectors past each other.
    The algorithm terminates with the controlled square matrix terms ordered by their mutually exclusive sectors in lexicographical order with respect to the order of their variables.
\end{proof}
\begin{remark}
    These procedures guarantee that controlled operators in mutually exclusive sectors are commutable past each other.
    This guarantee does not apply to controlled operators in the same sector, as this would require additional information about the operators themselves beyond being controlled operators.
\end{remark}

\subsection{Factorising Hamiltonians}
To present a small working example, we leverage both our rewrite rules for arithmetic ZW diagrams, and for controlled diagrams, to \emph{factor} them.  For example, for same-size square matrices $I, A, B$ and $a, b, c \in \mathbb{C}$:
\begin{gather}
    \tikzfig{tikz/poly/matspoly}
\end{gather}
Factoring Hamiltonians is important to optimise quantum algorithms for chemistry and physics simulations. However, previous graphical rewrites for factoring Hamiltonians had only been doable for Hamiltonians with concretely-specified matrix terms~\cite{shaikh2022sum}. This completeness result guarantees that for any Hamiltonian, even if its matrix terms are black-box, these graphical rewrite rules are capable of deriving any of its possible factorisations.

Recently, one of the new rules for controlled diagrams first proposed in this work, \eqref{rule:CMcpy}, was used in Ref.~\cite{mcdowallrose2025f2q} to derive a general form for any two-body fermionic Hamiltonian in second quantisation under any linear encoding.


