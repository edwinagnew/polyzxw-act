\section{Introduction}
Controlling or branching to different possible linear maps, relations, or channels is important across quantum information and quantum computation, and has been studied through many different approaches. In quantum algorithms common techniques are block encodings~\cite{gilyen2019qsvt, rall2020qalgsblockenc} and linear combination of unitaries~\cite{childs2012hamsimlcu}, while a number of formalisations have included routed quantum circuits~\cite{vanrietvelde2021routed}, the many-worlds calculus~\cite{chardonnet2023manyworlds}, categorifying signal flow diagrams~\cite{baez2015categoriesctrl}, and classical and quantum control in quantum modal logic~\cite{sati2023quantummonadology}.

The question we are interested in is how quantum graphical calculi such as the ZX~\cite{coecke2011zx}, ZW~\cite{coecke2010zw}, and ZH~\cite{backens2019zh} calculus can be augmented to support properties of quantum control.
An early use of controlled state diagrams was for proving constructive and rational angle ZX calculus completeness~\cite{jeandel2018zxconstructive}. More recently, controlled state and controlled matrix diagrams have been applied to addition and differentiation of ZX diagrams~\cite{jeandel2024adddiffzx}, differentiating and integrating ZX diagrams for quantum machine learning~\cite{wang2022diffintzx}, Hamiltonian exponentiation and simulation~\cite{shaikh2022sum}, and non-linear optical quantum computing~\cite{de2023light}. To sum ZX diagrams, these works have used controlled states along with the W generator from the ZW calculus.

Given how useful controlled diagrams are, a natural question to ask is why they work: What their underlying mathematical structures are, and which equational rewrites they satisfy.

First, we show that the set of all controlled $n$-partite states defines a commutative ring. We introduce $\boxplus$ which defines an Abelian group and $\boxtimes$ which defines a commutative monoid, and show that $\boxtimes$ distributes over $\boxplus$. The fragment of the qubit ZW calculus corresponding to controlled states, which we call \emph{arithmetic ZXW diagrams} hence defines a ring which we prove is isomorphic to multilinear polynomials $\mathbb{C}[x_1,...,x_n]/({x_1}^2,...,{x_n}^2)$, and prove completeness for. Analogously, we show that the set of all controlled square matrices on $n$ qubits defines a non-commutative ring $(\tilde{M^n}, \lowerbox[10]{\wspids}, \lowerbox[10]{\zspids})$.

We add controlled square matrices and rewrite rules for them to the ZXW-calculus, in which we plug controlled states into each control wire of controlled square matrices. We prove their completeness and that this is isomorphic to multivariate polynomials over same-size square matrices. Commutativity of controlled square matrices holds in the special case that the controls target mutually exclusive sectors, allowing copying of arbitrary controlled diagrams. As a result, we can factor multivariate polynomials over same-size square matrices. This means we now have the ability to factor any Hamiltonian in the ZXW-calculus~\cite{shaikh2022sum}, even with all its terms black-boxed.

Applying these findings, we present a quantum complexity theory result concerning efficient rewriting of quantum states to any arithmetic ZXW diagram, and prove that the higher-order map $\Control$ which sends square matrices to controlled square matrices satisfies some important algebraic properties.
In sum, these algebraic properties of quantum control give rise to powerful new graphical reasoning capabilities.