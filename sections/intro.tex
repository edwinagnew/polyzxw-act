\section{Introduction}


Examples of applications of category theory to quantum programs include to embed into a host functional programming language~\cite{rennela2020clctrllinlogic}, to do equational reasoning from algebraic theory~\cite{staton2015algqpl},


ZW calculus~\cite{coecke2011ghz}
completeness for arbitrary finite dimension~\cite{poor2023completeness}
applicatios to photonic quantum computing~\cite{de2023light}, quantum machine learning~\cite{koch2022quantum}, and Hamiltonian simulation~\cite{shaikh2022sum}.

Ref.~\cite{wilson2022diffpolycircml} presented Cartesian Distributive Categories exemplified by polynomial circuits, which are isomorphic to polynomials over arbitrary commutative semirings or rings.
They proved this isomorphism explicitly in Ref. for the case of Boolean circuit.

% We can add section references
In this paper, we first define a fragment of the qubit ZW calculus corresponding to controlled states. This defines a subcategory we prove is isomorphic to multilinear polynomials $\mathbb{C}[x_1,...,x_n]/({x_1}^2,...,{x_n}^2)$.

We then introduce a higher-order map $\Control$ which maps states to controlled states, and square matrices to controlled square matrices. We prove that $\Control$ is a lax monoidal functor on the subcategory $\Hilb_{\leq}$ of linear maps $n \rightarrow m$ such that $n \leq m$. We then apply the ZH calculus to show that multiple applications of $\Control$ i.e. multiple-controlling is monadic.

We show that the set of all controlled $n$-partite states defines a commutative ring $(\tilde{S^n},\boxplus,\boxtimes)$. We introduce $\boxplus$ which defines an Abelian group and $\boxtimes$ which defines a commutative monoid, and show that $\boxtimes$ distributes over $\boxplus$.

Analogously, we show that the set of all controlled square matrices on $n$ qubits defines a non-commutative ring $(\tilde{M^n},\scalebox{0.7}{\tikzfig{tikz/inline/W}},\scalebox{0.7}{\tikzfig{tikz/inline/Z}})$. Commutativity holds in the special case that the control conditions are mutually exclusive, allowing copying of arbitrary controlled diagrams. As a result, we can factor multivariate polynomials over same-size square matrices; this enables templated equational rewrites for factoring any Hamiltonians of the form in Ref.~\cite{shaikh2022sum}.



% functorial boxes in string diagrams~\cite{mellies2006functorial}
% polynomial factorisation~\cite{forbes2015complexity}


\section{Conclusion}

% Future work % qudits
We would like to try the very recently introduced mechanism of \emph{scoped effects}~\cite{lindley2024scoped} as an approach to better formulate the monadic nature of multiple-control.

We are also curious about reconciling the interpretation of diagrammatic differentiation of our arithmetic polynomial circuits by the approach in Ref.~\cite{wilson2023diffpolycirc}, with that of quantum circuits and ZX diagrams in Refs.~\cite{toumi2021diagdiff, wang2022diffintzx, jeandel2024adddiffzx}.