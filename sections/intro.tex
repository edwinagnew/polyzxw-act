\section{Introduction}
Controlling or branching to different possible linear maps, relations, or channels is important across quantum information and quantum computation, and has been studied through many different approaches. In quantum algorithms common techniques are block encodings~\cite{gilyen2019qsvt, rall2020qalgsblockenc} and linear combination of unitaries~\cite{childs2012hamsimlcu}, while a number of categorical formalisations have included routed quantum circuits~\cite{vanrietvelde2021routed}, the many-worlds calculus~\cite{chardonnet2023manyworlds}, categorifying signal flow diagrams~\cite{baez2015categoriesctrl}, and classical and quantum control in quantum modal logic~\cite{sati2023quantummonadology}.

The question we are interested in is how quantum graphical calculi such as the ZX~\cite{coecke2011zx}, ZW~\cite{coecke2010zw}, and ZH~\cite{backens2019zh} calculus can be augmented to support properties of quantum control.
An early use of controlled state diagrams was for proving constructive and rational angle ZX calculus completeness~\cite{jeandel2018zxconstructive}. More recently, controlled state and controlled matrix diagrams have been applied to addition and differentiation of ZX diagrams~\cite{jeandel2024adddiffzx}, differentiating and integrating ZX diagrams for quantum machine learning~\cite{wang2022diffintzx}, Hamiltonian exponentiation and simulation~\cite{shaikh2022sum}, and non-linear optical quantum computing~\cite{de2023light}. To sum ZX diagrams, these works have used controlled states along with the W generator from the ZW calculus.

Given how useful controlled diagrams are, a natural question to ask is why they work: What their underlying mathematical structures are, and which equational rewrites they satisfy.

In this paper, we first introduce a higher-order map $\Control$ which sends states to controlled states, and square matrices to controlled square matrices. We prove that $\Control$ is a lax monoidal functor on the subcategory of all square matrices. This allows us to use the functorial boxes of Ref.~\cite{mellies2006functorial} to control ZX diagrams so long as functoriality is only applied when the number of input and output wires are equal. Moreover, AND of the controls is a natural transformation corresponding to nested applications of $\Control$.

Next, we show that the set of all controlled $n$-partite states defines a commutative ring $(\tilde{S^n},\boxplus,\boxtimes)$. We introduce $\boxplus$ which defines an Abelian group and $\boxtimes$ which defines a commutative monoid, and show that $\boxtimes$ distributes over $\boxplus$. The fragment of the qubit ZW calculus corresponding to controlled states hence defines a subcategory we prove is isomorphic to multilinear polynomials $\mathbb{C}[x_1,...,x_n]/({x_1}^2,...,{x_n}^2)$. Analogously, we show that the set of all controlled square matrices on $n$ qubits defines a non-commutative ring $(\tilde{M^n}, \lowerbox[7]{\wspids}, \lowerbox[7]{\zspids})$.

We compose controlled states into each control wire of controlled square matrices to recover multivariate polynomials over same-size square matrices. Commutativity of controlled square matrices holds in the special case that the controls target mutually exclusive sectors, allowing copying of arbitrary controlled diagrams. As a result, we can factor multivariate polynomials over same-size square matrices. This means we can now factor any Hamiltonian in the ZXW calculus~\cite{shaikh2022sum}, even with all its terms black-boxed. In summary, these algebraic properties of quantum control give rise to powerful new graphical rewrite rules for black-box diagrams.