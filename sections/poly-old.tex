%%% This file is no longer in use in the main text. This Polynomials section has now been split into 3 sections: ring.tex, iso.tex, and ctrl.tex.
\section{Polynomials}
In this section, we reverse-engineer the underlying algebraic properties of controlled state and controlled square matrix diagrams. This builds up to diagrams for the unique normal form for states used for the first proofs of complete axiomatisation for qubit graphical calculi~\cite{hadzihasanovic2017thesis, Hadzihasanovic2018zwzxcomplete}.

All proofs in this section can be found in Appendix~\ref*{sec:appiso}.

\subsection{Rings}\label{sec:ring}
Let $\tilde{M_n}$ be the set of controlled square matrices on $n$ qubits. The goal of this section is to prove that the addition and multiplication operations introduced above induce a ring on $\tilde{M_n}$. Likewise, we show that the set of controlled $n$-qubit states $\tilde{S_n}$ also forms a ring. Before doing so, we prove a few important lemmas. The first lemma enables us to copy controlled matrices.

\begin{lemma}\label{lem:csq_copy}
    For any square matrix $M$, 
    \begin{equation}\label{eq:csq_copy}
    \tikzfig{tikz/con/csq_copy}
\end{equation}
\end{lemma}

Now we show that controlled matrix addition and multiplication satisfy the ring axioms. Associativity of $+, \times$ follow immediately from (\ref{rule:Aso}, \ref{rule:S1}), respectively. Commutativity of addition follows from the commutativity of matrix addition.

\begin{lemma}\label{lem:csq_add_comm}
    Let $M_1, M_2$ be $n \times n$ matrices. 
    \begin{equation}\label{eq:csq_add_comm}
        \tikzfig{tikz/con/csq_add_comm_statement}
    \end{equation}
\end{lemma}


The additive identity is defined as $\redzeroup \otimes I_n$:
\begin{equation*}
    \tikzfig{tikz/con/csq_add_id}
\end{equation*}


The multiplicative identity is defined very similarly as $\greenzeroup \otimes I_n$. The existence of additive inverses relies on the copying lemma from before.

\begin{lemma}
    The additive inverse of $\tilde{M}$ is $\raisebox{-5pt}{\numbergate[-1]} \circ \tilde{M} $
\end{lemma}
\begin{proof}
    \begin{equation*}
        \tikzfig{tikz/con/csq_add_inv}
    \end{equation*}
\end{proof}

Finally, we prove distributivity.

\begin{lemma}\label{lem:csq_dist}
    \begin{equation*}
        \tikzfig{tikz/con/csq_dist_statement}
    \end{equation*}
\end{lemma}


Combining the lemmas of this section shows that controlled matrices form a ring. A similar result can be shown for controlled states. Once again, we start with the ability to copy controlled states. 
\begin{lemma}\label{lem:cs_copy}
    For any state $\psi$,
    \begin{equation}\label{eq:cs_copy}
        \tikzfig{tikz/con/cs_copy}
    \end{equation}
\end{lemma}

Many of the ring axioms follow directly from basic ZXW rules. For example we can show commutativity of addition as follows:

\begin{lemma}\label{lem:cs_add_comm}
    For $n$-partite states $\psi_1, \psi_2$, $\tilde{\psi_1} \boxplus \tilde{\psi_2} = \tilde{\psi_2} \boxplus \tilde{\psi_1}  $
\end{lemma}

Associativity of $\boxplus$ follows similarly, using (\ref{rule:Aso}). Next we have the additive identity.

\begin{lemma}\label{lem:cs_add_id}
    $\tilde{\psi} \boxplus\mathbf{\tilde{0}} = \tilde{\psi}$
\end{lemma}


The additive inverse is defined similarly to the case of controlled matrices. 


\begin{lemma}\label{lem:cs_add_inv}
    For a controlled state $\tilde{\psi}$, its additive inverse is $\tilde{\psi} \circ \raisebox{-5pt}{\numbergate[-1]}$
\end{lemma}


Associativity and commutativity of $\boxtimes$ follow as before, using (\ref{rule:S1}) for $\lowerbox{\zspid}$. Finally, we must prove distributivity.


\begin{lemma}\label{lem:cs_dist}
    $\tilde{\psi_1} \boxtimes (\tilde{\psi_2} \boxplus \tilde{\psi_3}) = (\tilde{\psi_1} \boxtimes \tilde{\psi_2}) \boxplus (\tilde{\psi_1} \boxtimes \tilde{\psi_3})$
\end{lemma}

\begin{remark}
    A different addition and multiplication for controlled states was defined in Ref.~\cite{jeandel2018zxconstructive}. There corresponded to entry-wise addition and multiplication of statevectors, while our $\boxplus$ and $\boxtimes$ correspond to addition and multiplication of polynomials in bijective correspondence to controlled states, which we show next.
\end{remark}




\subsection{Arithmetic}

Its been known since 2011 that $\lowerbox{\wspid}, \lowerbox{\zspid}$ can be used to add and multiply number states $\numberstate$, respectively \cite{coecke2011ghz}. In the previous section we saw that $\lowerbox{\wspid}, \lowerbox{\zspid}$ can moreover be used to copy controlled diagrams. In this section, we explain this connection by demonstrating that controlled states are in fact isomorphic to multilinear polynomials. This being a bijection is a well-known folklore result in the study of entangled states, but to the best of our inquiries we are not aware of a proof. More generally, Ref.~\cite{wilson2023diffpolycirc} presented Cartesian Distributive Categories exemplified by polynomial circuits, which are isomorphic to polynomials over arbitrary commutative semirings or rings; their proof is non-constructive, giving explicit proof only for the case of Boolean circuits~\cite{wilson2021revderbool}.

Firstly, we describe how to interpret certain ZXW diagrams as polynomials. Consider the diagrams:
\begin{equation*}
    \tikzfig{tikz/poly/eg1}
\end{equation*}

If we treat the bottom wires as an indeterminate $x$, we can read these bottom-up as computing $x - 1$ and $2x + 3$, respectively. Moreover, since these diagrams are both controlled states, they can be added together,  yield a diagram resembling $3x + 2$:
\begin{equation*}
    \tikzfig{tikz/poly/eg4}
\end{equation*}

When trying to multiply these diagrams, rather than getting $(x-1)(2x+3) = 2x^2 + x - 3$, we instead get $x - 3$.
\begin{equation*}
    \tikzfig{tikz/poly/eg5}
\end{equation*}

The reason for the missing $2x^2$ term is that (\ref{eq:kill_quad}) implies $x^2 = 0$. Other than that, controlled state arithmetic appears to faithfully reflect polynomial arithmetic. To help formalise this correspondence, we introduce the following definition.

\begin{definition}
    A ZXW diagram with a single input on top is \textbf{arithmetic} if it contains only  $\lowerbox{\idwire}$, $\lowerbox{\swap}$ wires, $\lowerbox[10]{\wspids}$, $\lowerbox[10]{\zspids}$, $\lowerbox{\coWs}$ nodes and $\lowerbox{\numberstate}$ boxes.
\end{definition}
\begin{remark}
    This fragment of the ZXW-calculus defines a subcategory; adding $\lowerbox{\redzero}$, this is an instance of a Cartesian Distributive Category as defined in Ref.~\cite{wilson2023diffpolycirc}.
\end{remark}

To interpret an arithmetic ZXW diagram as an arithmetic expression, read $\lowerbox{\wspid}$ as $+$, $\lowerbox{\zspid}$ as $\times$, $\lowerbox{\numberstate}$ as the number $a$, $\lowerbox{\coW}$ as fanout, and output/bottom wires as variables $x_1, ..., x_n$ numbered from left to right. The following lemma establishes that all arithmetic diagrams are controlled states:
\begin{lemma}
    For any arithmetic diagram $A$, \begin{equation}\label{eq:arith_cs}
        \tikzfig{tikz/poly/arith0_statement}
    \end{equation}
\end{lemma}
\begin{proof}
    By definition, other than wires $A$ contains only $\raisebox{-7pt}{\wspids}$, $\raisebox{-7pt}{\zspids}$, $\raisebox{-5pt}{\coWs}$, and $\raisebox{-3pt}{\numberstate}$. All $\raisebox{-3pt}{\numberstate}$'s can be removed with (\ref{rule:Ept}). Meanwhile all the spiders copy $\raisebox{-5pt}{\redzero}$ due to (\ref{rule:Bs0}, \ref{rule:K0}, \ref{eq:wid}) respectively.
\end{proof}

Just as it is typical to represent a polynomial in normal form as a sum of products, it is possible to rewrite every arithmetic diagram into a normal form as a single $\lowerbox{\wspids}$, followed by a layer of $\lowerbox{\zspids}$, followed by a layer of $\numberstate, \lowerbox{\coWs}$. 

\begin{definition}
    An $n$-output arithmetic diagram is said to be written in \textbf{polynormal form} (PNF) if it looks like:
    \begin{equation*}
        \tikzfig{tikz/poly/pnf}
    \end{equation*}

    The $i$th coefficient $a_i$ is connected to the $k$th $\lowerbox{\coWs}$ iff the $k$th bit in the binary expansion of $i$ is 1. 
\end{definition}

This normal form is familiar from completeness of all linear maps for qubits~\cite{Hadzihasanovic2018zwzxcomplete} and for qudits~\cite{poor2023completeness}. The reason we introduce the definition of a PNF is that it is an arithmetic diagram and therefore has a more immediate arithmetic interpretation. The reason for the specific connectivity condition is that it enables a PNF to directly represent its own matrix.

\begin{prop}\label{prop:vec_pnf}
    \begin{gather}\label{eq:pnf_vec}
        \tikzfig{tikz/poly/pnf} ~=~ 
        \begin{bmatrix}
            1 &  a_0 \\ 0 & a_1 \\ ... & ... \\ 0 & a_{2^n-1}
        \end{bmatrix}
    \end{gather}
\end{prop} 

\begin{proof}
    See Appendix~\ref*{sec:appiso}.
\end{proof}

Thus, every controlled state can be represented as at least one arithmetic diagram (namely, its PNF). Moreover, we now show that any other arithmetic diagram can always be rewritten to its PNF.


\begin{thm}\label{thm:uni_pnf}
    All arithmetic diagrams can be rewritten into PNF through application of ZXW rules. Therefore, those ZXW-calculus rules applied suffice for completeness for the arithmetic fragment of the ZXW-calculus.
\end{thm}

\begin{proof}
    An algorithm to rewrite any arithmetic diagram to PNF is given in Appendix~\ref*{sec:appiso}.
\end{proof}

\subsection{Isomorphism}

At last we can prove the isomorphism. Throughout we shall let $\polyring$ denote the ring $\mathbb{C}[x_1, ..., x_{n}]/(x_1^2, ..., x_{n}^2)$.


\begin{thm}\label{thm:iso}
    There is an isomorphism $\polyring \simeq \tilde{S_n}$
\end{thm}

First, we shall define the map $\phi_n: \polyring \to \csring$ before proving it induces an isomorphism. $\phi_n$ is defined to map an arbitrary polynomial $p(x_1, ..., x_n) = a_0 + a_1x_n + ... + a_{2^n-1}x_1x_2...x_n$ to the following PNF:
    \begin{equation*}
        \phi_n(p) ~=~ \tikzfig{tikz/poly/pnf}
    \end{equation*}

Some important special cases are mapping scalars $a \in \mathbb{C}$:
 \begin{equation*}
        \tikzfig{tikz/poly/homproof/homdef1}
\end{equation*}

And mapping indeterminates $x_i$:
    \begin{equation*}
        \tikzfig{tikz/poly/homproof/homdef2}
    \end{equation*}

The full proof is found in Appendix~\ref*{sec:appiso}.

\subsection{Factorisation}
Instead of the indeterminates being complex numbers represented by $\numberstate$'s, we can let them be same-size matrices represented by controlled square matrix diagrams.
We then have that:
\begin{thm}
    ZXW diagrams where the outputs of an arithmetic ZW diagram are each plugged into controls of same-size controlled matrices, are isomorphic to multivariate polynomials over same-size square matrices with complex number coefficients.
    The rules for their completeness are the same subset of ZXW rules used for completeness for arithmetic diagrams in the ZXW-calculus in \Cref{thm:uni_pnf}, plus the controlled square matrix as a generator along with the four rewrite rules for it in \Cref{def:ctrlsqmat}, \Cref{lem:csq_copy}, and \Cref{lem:csq_add_comm}.
\end{thm}
\begin{proof}
    The proof is by the same algorithm for rewriting to PNF as \Cref{thm:uni_pnf}, modifying step (6) to copy controlled square matrices using \Cref{lem:csq_copy}, using \Cref{lem:csq_add_comm} to commute controlled square matrices whose controls act on mutually exclusive sectors.
\end{proof}

As an application, we leverage both our rewrite rules for arithmetic ZW diagrams, and for controlled diagrams, to \emph{factor} them.  For example, for same size square matrices $I, A, B$ and $a, b, c \in \mathbb{C}$:
\begin{gather*}
    \tikzfig{tikz/poly/matspoly}
\end{gather*}
Factoring Hamiltonians is important to optimise quantum algorithms for chemistry and physics simulations. However, previous graphical rewrites for factoring Hamiltonians had only been doable for Hamiltonians with concretely-specified matrix terms~\cite{shaikh2022sum}. This completeness result guarantees that for any Hamiltonian, even if its matrix terms are black-box, these graphical rewrite rules are capable of deriving any of its possible factorisations.