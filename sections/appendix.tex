\section{Basic Lemmas}\label{sec:applem}


\begin{lemma}
    \begin{equation}\label{eq:xcpy}
        \tikzfig{tikz/lemmas/xcpy_statement}
    \end{equation}
    \end{lemma}
    \begin{proof}
        \begin{equation*}
        \tikzfig{tikz/lemmas/xcpy}
    \end{equation*}
    \end{proof}
  
  
  
  \begin{lemma}
      \begin{equation}\label{eq:zerobox}
          \tikzfig{tikz/lemmas/zerobox_statement}
      \end{equation}
  \end{lemma}
  \begin{proof}
      \begin{equation*}
          \tikzfig{tikz/lemmas/zerobox}
      \end{equation*}
  \end{proof}
  
  \begin{lemma}
      \begin{equation}\label{eq:cp_add}
          \tikzfig{tikz/lemmas/cpadd_statement}
      \end{equation}
  \end{lemma}
  \begin{proof}
      \begin{equation*}
      \tikzfig{tikz/lemmas/cpadd}
  \end{equation*}
  \end{proof}
  
  \begin{lemma}
    \begin{equation}\label{eq:x-x=0}
      \tikzfig{tikz/lemmas/cp_add2_statement}
    \end{equation}
  \end{lemma}
  \begin{proof}
    \begin{equation*}
        \tikzfig{tikz/lemmas/cp_add2}
    \end{equation*}
  \end{proof}
  
  
  \begin{lemma}
    \begin{equation}\label{eq:cpk_add}
      \tikzfig{tikz/lemmas/cpk_add_statement}
    \end{equation}
  \end{lemma}
  \begin{proof}
    \begin{equation*}
      \tikzfig{tikz/lemmas/cpk_add}
    \end{equation*}
  \end{proof}
  
  
  \begin{lemma}{}{}
    \begin{equation}\label{eq:kill_quad}
    \tikzfig{tikz/lemmas/kill_quad_statement}
  \end{equation}
  \end{lemma}
  \begin{proof}
    \begin{equation*}
        \tikzfig{tikz/lemmas/kill_quad}
    \end{equation*}
  \end{proof}
  
  
  \begin{lemma}
    \begin{equation}\label{eq:dbl_dist}
    \tikzfig{tikz/lemmas/dbl-dist-statement}
  \end{equation}
  \end{lemma}
  \begin{proof}
    \begin{equation*}
    \tikzfig{tikz/lemmas/dbl-dist}
  \end{equation*}
  \end{proof}
  
  \begin{lemma}
    \begin{equation}\label{eq:dist_circ}
      \tikzfig{tikz/lemmas/dist_statement}
    \end{equation}
  \end{lemma}
  \begin{proof}
    \begin{equation*}
      \tikzfig{tikz/lemmas/dist}
    \end{equation*}
  \end{proof}
  
  
  %TODO: find proofs!
  
  %\begin{equation}\label{eq:wid}
  %    \tikzfig{tikz/lemmas/wid}
  %\end{equation}
  
  
  %\begin{equation}\label{eq:wont}
  %  \tikzfig{tikz/lemmas/wont}
  %\end{equation}
  
  
  
  
  
  \begin{lemma}
    \begin{equation}\label{eq:0times}
        \tikzfig{tikz/lemmas/0times_statement}
    \end{equation}
  \end{lemma}
  \begin{proof}
    \begin{equation*}
        \tikzfig{tikz/lemmas/0times}
    \end{equation*}
  \end{proof}

\section{Proofs}\label{sec:appiso}

Proof of proposition \ref*{prop:vec_pnf}

\begin{proof}
    We prove by induction on $n$.
    
    For the base case, $n=0$. The only PNF with no outputs is a number so we have: $$\numberstate[a_0] = \begin{bmatrix}
        1 & a_0
    \end{bmatrix}$$ as desired.
    
    For inductive hypothesis, we assume that (\ref{eq:pnf_vec}) holds for every PNF on $n$ outputs. We use this hypothesis to extend it to PNFs with $n+1$ outputs. 
    
    Let $D$ be an arbitrary PNF with $n+1$ outputs. Firstly, observe that $x_{n+1}$ is connected to only the odd coefficients $\{a_{2k+1}\}$ since these are exactly the indices with $1$ in the least significant bit. Thus we can rewrite:
    \begin{gather*}
        \raisebox{-20pt}{\ddiag} ~=~ \tikzfig{tikz/poly/lemmas/uni10} \neweqline ~=~ \tikzfig{tikz/poly/lemmas/uni11} \neweqline ~\eqq{\ref{rule:BZW}}~ \tikzfig{tikz/poly/lemmas/uni12} \neweqline 
        ~=~ \tikzfig{tikz/poly/lemmas/uni13}
    \end{gather*}
    Where $D_{even}, D_{odd}$ are PNF diagrams. Since they are over $n$ variables, we can apply the inductive hypothesis and obtain:
    \begin{equation}\tag{*}
        D_{even} = \begin{bmatrix}
            1 & a_0 \\ 0 & a_2 \\ ... & ... \\ 0 & a_{2^{n+1}-2}
        \end{bmatrix}, 
        D_{odd} = \begin{bmatrix}
            1 & a_1 \\ 0 & a_3 \\ ... & ... \\ 0 & a_{2^{n+1}-1}
        \end{bmatrix}
    \end{equation}


    Next, plugging red we observe:
    \begin{equation*}
        \tikzfig{tikz/poly/lemmas/uni2}
    \end{equation*}
    Meanwhile,
    \begin{equation*}
        \tikzfig{tikz/poly/lemmas/uni3}
    \end{equation*}

    Summing these together,
    \begin{gather*}
        \tikzfig{tikz/poly/lemmas/uni4}
        \neweqline ~=~ (D_{even} \otimes \ketz) + (D_{odd}\keto \brao \otimes  \keto) \neweqline \eqq{*}~ \begin{bmatrix}
            1 & a_0 \\ 0 & a_2 \\ ... & ... \\ 0 & a_{2^{n+1}-2}
        \end{bmatrix} \otimes \ketz + \begin{bmatrix}
            0 & a_1 \\ 0 & a_3 \\ ... & ... \\ 0 & a_{2^{n+1}-1}
        \end{bmatrix} \otimes \keto  \neweqline
        ~=~ \begin{bmatrix}
            1 & a_0 \\ 0 & 0 \\ 0 & a_2 \\ 0 & 0 \\ ... & ... \\ 0 & a_{2^{n+1}-2} \\ 0 & 0
        \end{bmatrix} + \begin{bmatrix}
            0 & 0 \\ 0 & a_1 \\ 0 & 0 \\ 0 & a_3 \\ ... & ... \\ 0 & 0 \\ 0 & a_{2^{n+1}-1}
        \end{bmatrix} 
        ~=~ \begin{bmatrix}
            1 & a_0 \\ 0 & a_1 \\ 0 & a_2 \\ 0 & a_3 \\ ... & ... \\ 0 & a_{2^{n+1}-2} \\ 0 & a_{2^{n+1} -1}
        \end{bmatrix}
    \end{gather*}

    Completing the inductive step.
\end{proof}

\subsection{Isomorphism}

Proof of Theorem \ref*{thm:iso}
\begin{proof}
    
    First, we show $\phi_n$ is a homomorphism, i.e. \begin{equation*}
        \forall p, q \in \polyring, \phi_n(p + q) = \phi_n(p) \boxplus \phi_n(q) ,\quad \phi_n(p \times q) = \phi_n(p) \boxtimes \phi_n(q)
    \end{equation*} The strategy for the proof will be an induction on $n$. 

    \medskip
    
    \textbf{Base case:}
    We have not defined controlled states for $n=0$, so the base case begins with $n=1$.
    Let $p, q \in \polyring[1]$. Write as $p(x_1) = a_0 + a_1x_1, q(x_1) = b_0 + b_1x_1$, where $a_0, a_1, b_0, b_1 \in \mathbb{C}$. Then since $p + q = a_0 + b_0 + (a_1 + b_1)x_1$,
    \begin{equation*}
        \tikzfig{tikz/poly/homproof/hombaseadd}
    \end{equation*}

    Meanwhile, since $p \times q = a_0a_1 + (a_0b_1 + a_1b_0)x_1$, 
    \begin{gather*}
        \phi_1(p) \boxtimes \phi_1(q) ~=~ \tikzfig{tikz/poly/homproof/bt1} ~\eqq{\ref{eq:dbl_dist}}~ \tikzfig{tikz/poly/homproof/bt2} \neweqline 
        ~\eqq{\ref{eq:dist_circ}}~ \tikzfig{tikz/poly/homproof/bt3} ~\eqq{\ref{rule:Pcy}}~ \tikzfig{tikz/poly/homproof/bt4} \neweqline
        ~\eqq{\ref{eq:kill_quad}}~ \tikzfig{tikz/poly/homproof/bt5} ~\eqq{\ref{eq:0times}}~ \tikzfig{tikz/poly/homproof/bt6} ~\eqq{\ref{eq:cp_add}}~ \tikzfig{tikz/poly/homproof/bt7} \neweqline ~= \phi_1(p \times q)
    \end{gather*}


    Completing the base case.

    \medskip


    \textbf{Inductive step:}

    Let $Hom(n)$ assert than $\phi_n$ is a homomorphism.  Then for the inductive step we wish to prove that $\forall n, Hom(n) \implies Hom(n+1)$.

    The proof relies on the recursive definition of $R[x_1, x_2] = R[x_1][x_2]$, for any ring $R$, to rewrite an arbitrary polynomial $p(x_1, ..., x_{n+1}) = a_0 + a_1x_{n+1} + ... + a_{2^{n+1}-1}x_1x_2...x_{n+1} \in \polyring[n+1]$ as $p(x_{n+1}) = p_0 + p_1x_{n+1}$, where $p_0, p_1 \in \polyring$. This allows the $p_i$ to be treated similarly to the scalars in the base case. To emphasise this, they will be drawn in green boxes. To help distinguish when an operation is covered by the inductive hypothesis, the wires for variables $x_1, ..., x_n$ will be drawn in light blue, while the $x_{n+1}$ wires will be drawn in black. Thus the inductive hypothesis states that:
    \begin{equation}\label{eq:ih1}\tag{IH1}
        \tikzfig{tikz/poly/homproof/ih1}
    \end{equation}
    \begin{equation}\label{eq:ih2}\tag{IH2}
        \tikzfig{tikz/poly/homproof/ih2}
    \end{equation}

    Let $p(x_{n+1}) = p_0 + p_1x_{n+1}, q(x_{n+1}) = q_0 + q_1x_{n+1}$, where $p_0, p_1, q_0, q_1 \in \polyring$. Then for addition:
    \begin{gather*}
        \phi_{n+1}(p) \boxplus \phi_{n+1}(q) ~=~ \tikzfig{tikz/poly/homproof/sa1} ~\eqq{\ref{rule:Aso}}~ \tikzfig{tikz/poly/homproof/sa2} \neweqline
        ~ \eqq{IH1} ~ \tikzfig{tikz/poly/homproof/sa3} ~\eqq{\ref{rule:BZW}}~ \tikzfig{tikz/poly/homproof/sa4} ~\eqq{IH1}~ \tikzfig{tikz/poly/homproof/sa5} \neweqline
        ~=~ \phi_{n+1}(p_0 + q_0 + (p_1 + q_1)x_{n+1}) ~=~ \phi_{n+1}(p + q)
    \end{gather*}


    Similarly, for multiplication:
    \begin{gather*}
    \phi_{n+1}(p) \boxtimes \phi_{n+1}(q) ~=~ \tikzfig{tikz/poly/homproof/st1} \eqq{\ref{eq:dbl_dist}}  ~ \tikzfig{tikz/poly/homproof/st2} \neweqline
    \eqq{\ref{eq:cs_copy}} ~ \tikzfig{tikz/poly/homproof/st3} \eqq{\ref{eq:ih2}} ~ \tikzfig{tikz/poly/homproof/st4} \neweqline
    \eqq{\ref{rule:BZW}} ~ \tikzfig{tikz/poly/homproof/st5} ~ = ~ \tikzfig{tikz/poly/homproof/st6} \neweqline
     \eqq{\ref{eq:cs_copy}} ~ \tikzfig{tikz/poly/homproof/st7} ~ \eqq{\ref{eq:ih2}} ~ \tikzfig{tikz/poly/homproof/st8} \neweqline
    \eqq{\ref{rule:BZW}} ~ \tikzfig{tikz/poly/homproof/st9} ~ = ~ \tikzfig{tikz/poly/homproof/st10} \neweqline
    \eqq{\ref{eq:kill_quad}} ~ \tikzfig{tikz/poly/homproof/st11} ~ \eqq{\ref{eq:arith_cs}, \ref{eq:wid}} ~ \tikzfig{tikz/poly/homproof/st12} \neweqline ~ \eqq{\ref{eq:ih2}} ~ \tikzfig{tikz/poly/homproof/st13}
      ~ \eqq{\ref{rule:BZW}} ~ \tikzfig{tikz/poly/homproof/st14} ~ \eqq{\ref{eq:ih1}} ~ \tikzfig{tikz/poly/homproof/st15} \neweqline
     ~=~ \phi_{n+1}(p_0q_0 + (p_0q_1 + p_1q_0)x_{n+1} )
     ~=~ \phi_{n+1}(p \times q)
    \end{gather*}

    \medskip

    This completes the inductive step, proving that $\forall n > 1$, $\phi_n$ is a homomorphism.

    \bigskip

    Finally, to see $\phi_n$ is an isomorphism, we use proposition \ref{prop:uni_pnf} to write an arbitrary controlled state in PNF:
    \begin{gather*}
        \begin{bmatrix}
            1 & a_0 \\ 0 & a_1 \\ ... & .. \\ 0 & a_{2^{n}-1}
        \end{bmatrix}
        = \tikzfig{tikz/poly/pnf}
    \end{gather*}

    Then all we have to do is interpret it as the image of a polynomial:
    \begin{gather*}
        \tikzfig{tikz/poly/pnf} ~=~ \tikzfig{tikz/poly/homproof/iso4} \neweqline ~=~ \phi_{n}(a_0) + \phi_{n}(a_1x_{n}) + ... + \phi_{n}(a_{2^{n}-1x_1x_2...x_{n}}) \neweqline ~=~ \phi_{n}(a_0 + a_1x_{n} + ... + a_{2^{n}-1}x_1x_2...x_{n}) 
    \end{gather*}
\end{proof}