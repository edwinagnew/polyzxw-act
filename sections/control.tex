\section{Controlled Diagrams}

\subsection{Definitions}

As defined in \cite{shaikh2022sum}, 

\begin{definition}
    For an arbitrary square matrix $D$, the controlled matrix of $D$ is the diagram $\tilde{D}$ such that:

    \begin{equation}
        \tikzfig{tikz/con/c_mat1}
    \end{equation} 

    \begin{equation}
        \tikzfig{tikz/con/c_mat2}
    \end{equation} 
\end{definition}

It is possible to perform matrix arithmetic with controlled diagrams. 

\begin{prop}
    Given controlled matrices $\tilde{M_1}, ..., \tilde{M_k}$, the controlled matrix $\widetilde{\Pi_i M_i}$ is given by
    \begin{equation*}
        \tikzfig{tikz/con/cm_prod}
    \end{equation*}

    Given controlled matrices $\tilde{M_1}, ..., \tilde{M_k}$ and complex numbers $c_1, ..., c_k$, the controlled matrix $\widetilde{\Sigma_i c_i M_i}$ is given by
    \begin{equation*}
        \tikzfig{tikz/con/cm_sum}
    \end{equation*}
\end{prop}

\begin{proof}
    See propositions 3.3 and 3.4 in \cite{shaikh2022sum} 
\end{proof}

We can also defined the analogue for states

\begin{definition}
    For an arbitrary state $\psi$, the controlled state of $\psi$ is the diagram $\tilde{\psi}$ such that:

    \begin{equation}\label{eq:cstate}
        \tikzfig{tikz/con/c_state}
    \end{equation}
\end{definition}

The addition and multiplication of controlled states are defined similarly to controlled matrix arithmetic, except that a layer of $\lowerbox{\coW}$s are appended at the bottom to preserve the number of outputs.

\begin{equation*}
    \tikzfig{tikz/con/cs_add_def} \qquad         \tikzfig{tikz/con/cs_times_def}
\end{equation*}

The role of $\lowerbox{\coW}$ is to \textit{copy} inputs, as shown in the next subsection.

\subsection{Functor}

The operation of turning a square matrix to its controlled diagram can be made into a lax monoidal functor $F: \mathbf{EndVect} \to \mathbf{Vect}$, where $\mathbf{EndVect}$ is the category of vector space endomorphisms (i.e. square matrices). An additional horizontal wire is required to facilitate composition. Let $D \in Hom_{EndVect}(V, V)$.

\begin{equation}\label{eq:F_def}
    F:: \tikzfig{tikz/func/F_def}
\end{equation}

In the functorial box notation of \cite{mellies2006functorial}, this would be:
\begin{equation}
    \tikzfig{tikz/func/F_def_box}
\end{equation}


\begin{prop}
    The map $F$ defined in (\ref{eq:F_def}) is a lax monoidal funtor.
\end{prop}
\begin{proof}
    On $id_V: V \to V$:
    \begin{equation*}
        \tikzfig{tikz/func/F_id}
    \end{equation*}

    Composing $F(D_2) \circ F(D_1)$:
    \begin{equation*}
        \tikzfig{tikz/func/F_comp}
    \end{equation*}

    Where $(*)$ follows from
    \begin{equation*}
        \tikzfig{tikz/func/F_comp2}
    \end{equation*}

    $F$ preserves the monoidal unit since $\mone_{\mathbf{EndVect}} = \mone_{\mathbf{Vect}} = \lowerbox{\emptydiag}$

    $F$ is lax thanks to the following structure morphism: $\phi_{V, W}: F(V) \otimes F(W) \to F(V \otimes W)$:
    \begin{equation*}
        \phi_{V, W} = \tikzfig{tikz/func/F_lax_def}
    \end{equation*}

    $\phi$ is natural since for any $D_1: V \to V, D_2: W \to W$, we have:
    \begin{equation*}
        \tikzfig{tikz/func/F_lax_nat}
    \end{equation*}


    Finally, $\phi$ satisfies the coherence condition since for any $U, V, W$:
    \begin{equation*}
        \tikzfig{tikz/func/F_lax_ass}
    \end{equation*}


    
\end{proof}

\subsection{Monad?}

