\section{Controlled Diagrams}

\subsection{Definitions}

As defined in \cite{shaikh2022sum}, 

\begin{definition}
    For an arbitrary square matrix $D$, the controlled matrix of $D$ is the diagram $\tilde{D}$ such that:

    \begin{equation}
        \tikzfig{tikz/con/c_sq1}
    \end{equation} 

    \begin{equation}
        \tikzfig{tikz/con/c_sq2}
    \end{equation} 
\end{definition}

It is possible to perform matrix arithmetic with controlled diagrams. 

\begin{prop}
    Given controlled matrices $\tilde{M_1}, ..., \tilde{M_k}$, the controlled matrix $\widetilde{\Pi_i M_i}$ is given by
    \begin{equation*}
        \tikzfig{tikz/con/csq_prod}
    \end{equation*}

    Given controlled matrices $\tilde{M_1}, ..., \tilde{M_k}$ and complex numbers $c_1, ..., c_k$, the controlled matrix $\widetilde{\Sigma_i c_i M_i}$ is given by
    \begin{equation*}
        \tikzfig{tikz/con/csq_sum}
    \end{equation*}
\end{prop}

\begin{proof}
    See propositions 3.3 and 3.4 in \cite{shaikh2022sum} 
\end{proof}

We can also defined the analogue for states

\begin{definition}
    For an arbitrary state $\psi$, the controlled state of $\psi$ is the diagram $\tilde{\psi}$ such that:

    \begin{equation}\label{eq:cstate}
        \tikzfig{tikz/con/c_state}
    \end{equation}
\end{definition}

The addition and multiplication of controlled states are defined similarly to controlled matrix arithmetic, except that a layer of $\lowerbox{\coW}$s are appended at the bottom to preserve the number of outputs.

\begin{equation*}
    \tikzfig{tikz/con/cs_add_def} \qquad         \tikzfig{tikz/con/cs_times_def}
\end{equation*}

The role of $\lowerbox{\coW}$ is to \textit{copy} inputs, as shown in section \ref{sec:ring}.

Combining the notions of controlled square matrices and controlled states, we can a more general notion of controlled matrices. 

\begin{definition}
    For all $m \leq n$, the controlled matrix of an arbitrary matrix $M \in \mathbb{C}^{m \times n}$ is defined as the diagram $\tilde{M}$ such that:
    \begin{equation*}
        \tikzfig{tikz/con/c_mat1} \qquad \tikzfig{tikz/con/c_mat2}
    \end{equation*}
\end{definition}

We focus on matrices with non-decreasing dimension to avoid cases like the following which fail to satisfy funtoriality:
\begin{equation*}
    \tikzfig{tikz/con/cm_fail}
\end{equation*} 

\subsection{Functor}

The operation of turning a non-dimension-decreasing matrix to its controlled diagram can be made into a lax monoidal functor. Let $\mathbf{Hilb_{\leq}}$ be the subcategory of Hilbert spaces and non-dimension-descreasing linear transformations. Adding an additional horizontal wire to facilitate composition, $F: \mathbf{Hilb_{\leq}} \to \mathbf{Hilb}$ is defined as follows for arbitrary  $D \in Hom_{Hilb_{\leq}}(V, W)$.

\begin{equation}\label{eq:F_def}
    F:: \tikzfig{tikz/func/F_def}
\end{equation}

In the functorial box notation of \cite{mellies2006functorial}, this would be:
\begin{equation}
    \tikzfig{tikz/func/F_def_box}
\end{equation}


\begin{prop}
    The map $F$ defined in (\ref{eq:F_def}) is a lax monoidal funtor.
\end{prop}
\begin{proof}
    On $id_V: V \to V$:
    \begin{equation*}
        \tikzfig{tikz/func/F_id}
    \end{equation*}

    Let $D_1: U \to V$, $D_2: V \to W$, where $U, V, W$ have dimensions $l, m, n$ respectively. Then composing $F(D_2) \circ F(D_1)$:
    \begin{equation*}
        \tikzfig{tikz/func/F_comp}
    \end{equation*}

    Where $(*)$ follows from
    \begin{equation*}
        \tikzfig{tikz/func/F_comp2}
    \end{equation*}
    \begin{equation*}
        \tikzfig{tikz/func/F_comp3}
    \end{equation*}

    $F$ preserves the monoidal unit since $\mone_{\mathbf{Hilb_{\leq}}} = \mone_{\mathbf{Hilb}} = \lowerbox{\emptydiag}$

    $F$ is lax thanks to the following structure morphism: $\phi_{V, W}: F(V) \otimes F(W) \to F(V \otimes W)$:
    \begin{equation*}
        \phi_{V, W} = \tikzfig{tikz/func/F_lax_def}
    \end{equation*}

    $\phi$ is natural since for any $D_1: V \to V', D_2: W \to W'$, we have:
    \begin{equation*}
        \tikzfig{tikz/func/F_lax_nat}
    \end{equation*}


    Finally, $\phi$ satisfies the coherence condition since for any $U, V, W$:
    \begin{equation*}
        \tikzfig{tikz/func/F_lax_ass}
    \end{equation*}


    
\end{proof}

\subsection{Monad}

This section proves that controlling a controlled diagram gives the AND of the control wires, thus yielding a monad. First show how to represent the binary AND gate in the ZXW calculus.
\begin{equation*}
    \tikzfig{tikz/func/and_def}
\end{equation*}
We can verify this computes the AND gate by computing on basis states.
\begin{equation}\label{eq:and1}
    \tikzfig{tikz/func/and1}
\end{equation}
Thus $AND(1, x) = x$. Since the diagram is clearly commutative, it remains to check $AND(0, 0) = 0$.
\begin{equation}\label{eq:and00}
    \tikzfig{tikz/func/and00}
\end{equation}


\begin{prop}
    \begin{equation*}
        \tikzfig{tikz/func/FF_statement}
    \end{equation*}
\end{prop}

\begin{proof}
    Plugging basis states:
    \begin{equation*}
        \tikzfig{tikz/func/FF1}
    \end{equation*}
    \begin{equation*}
        \tikzfig{tikz/func/FF2}
    \end{equation*}
\end{proof}

In order to make multiple-control a monad, we first need to make $F$ an endofunctor. Although $F$ does take DND matrices to DND matrices, the definition of a monad requires that the components of the multiplication $\mu_V: F^2 V \to FV$ live in the codomain. However, AND is a dimension decreasing matrix so does not live in $\mathbf{Hilb_{\leq}}$. To get around this, we define a new category $\mathbf{Hilb_j}$ consisting of square matrices on the bottom $j$ wires and control logic on the remaining wires.  

Still functor

Now define $F_V: F^2 V \to FV$ and $\eta_V: id_V \to FV$ as follws:
\begin{equation*}
    \tikzfig{tikz/func/mu_def} \qquad \tikzfig{tikz/func/eta_def}
\end{equation*}

\begin{prop}
    $(F, \mu, \eta)$ defined a monad on $\mathbf{Hilb_j}$
\end{prop}
\begin{proof}
    Firstly, $\mu$ is a natural transformation since for any $M: V \to V$:
    \begin{equation*}
        \tikzfig{tikz/func/mu_nat}
    \end{equation*}


    Similary, $\eta$ is a natural transformation since for any $M: V \to V$:
    \begin{equation*}
        \tikzfig{tikz/func/eta_nat}
    \end{equation*}

    For the unitality coherence condition:
    \begin{equation*}
        \tikzfig{tikz/func/coh1}
    \end{equation*}




\end{proof}