\section{Controlled Diagrams}

\subsection{Definitions}

We recall the definition of a controlled diagram from \cite{shaikh2022sum}, 

\begin{definition}
    For an arbitrary square matrix $M$, the controlled matrix of $M$ is the diagram $\tilde{M}$ such that:

    \begin{equation}
        \tikzfig{tikz/con/c_sq1}
    \end{equation} 

    \begin{equation}
        \tikzfig{tikz/con/c_sq2}
    \end{equation} 
\end{definition}

It is possible to perform matrix arithmetic with controlled diagrams. 

\begin{prop}
    Given controlled matrices $\tilde{M_1}, ..., \tilde{M_k}$, the controlled matrix $\widetilde{\Pi_i M_i}$ is given by
    \begin{equation*}
        \tikzfig{tikz/con/csq_prod}
    \end{equation*}

    Given controlled matrices $\tilde{M_1}, ..., \tilde{M_k}$ and complex numbers $c_1, ..., c_k$, the controlled matrix $\widetilde{\Sigma_i c_i M_i}$ is given by
    \begin{equation*}
        \tikzfig{tikz/con/csq_sum}
    \end{equation*}
\end{prop}

\begin{proof}
    See propositions 3.3 and 3.4 in \cite{shaikh2022sum} 
\end{proof}

We can also defined the analogue for states.

\begin{definition}
    For an arbitrary state $\psi$, the controlled state of $\psi$ is the diagram $\tilde{\psi}$ such that:

    \begin{equation}\label{eq:cstate}
        \tikzfig{tikz/con/c_state}
    \end{equation}
\end{definition}

The addition and multiplication of controlled states are defined similarly to controlled matrix arithmetic, except that a layer of $\lowerbox{\coW}$s are appended at the bottom to preserve the number of outputs.

\begin{equation*}
    \tikzfig{tikz/con/cs_add_def} \qquad         \tikzfig{tikz/con/cs_times_def}
\end{equation*}

The role of $\lowerbox{\coW}$ is to \textit{copy} inputs, as shown in section \ref{sec:ring}.

%Combining the notions of controlled square matrices and controlled states, we can a more general notion of controlled matrices. 

%\begin{definition}
%    For all $m \leq n$, the controlled matrix of an arbitrary matrix $M \in \mathbb{C}^{m \times n}$ is defined as the diagram $\tilde{M}$ such that:
%    \begin{equation*}
%        \tikzfig{tikz/con/c_mat1} \qquad \tikzfig{tikz/con/c_mat2}
%    \end{equation*}
%\end{definition}

%We focus on matrices with non-decreasing dimension to avoid cases like the following which fail to satisfy funtoriality:
%\begin{equation*}
%    \tikzfig{tikz/con/cm_fail}
%\end{equation*} 

\subsection{Control Functor}

%The operation of turning a non-dimension-decreasing matrix to its controlled diagram can be made into a lax monoidal functor. Let $\mathbf{Hilb_{\leq}}$ be the subcategory of Hilbert spaces and non-dimension-descreasing linear transformations. Adding an additional horizontal wire to facilitate composition, $F: \mathbf{Hilb_{\leq}} \to \mathbf{Hilb}$ is defined as follows for arbitrary  $D \in Hom_{Hilb_{\leq}}(V, W)$.
The operation of turning a square matrix to its controlled diagram can be made into a lax monoidal functor. Let $\mathbf{Hilb_{sq}}$ be the subcategory of Hilbert spaces and square matrices. Adding an additional horizontal wire to facilitate composition, $\Control: \mathbf{Hilb_{sq}} \to \mathbf{Hilb}$ is defined as follows for arbitrary  $M \in Hom_{Hilb_{sq}}(V, V)$.

\begin{equation}\label{eq:F_def}
    \Control :: \tikzfig{tikz/func/F_def}
\end{equation}

In the functorial box notation of \cite{mellies2006functorial}, this would be:
\begin{equation}
    \tikzfig{tikz/func/F_def_box}
\end{equation}


\begin{prop}
    The map $\Control$ defined in (\ref{eq:F_def}) is a lax monoidal funtor.
\end{prop}
\begin{proof}
    On $id_V: V \to V$:
    \begin{equation*}
        \tikzfig{tikz/func/F_id}
    \end{equation*}

    Let $M_1: U \to V$, $M_2: V \to W$. Then composing $\Control(M_2) \circ \Control(M_1)$:
    \begin{equation*}
        \tikzfig{tikz/func/F_comp}
    \end{equation*}

    Where $(*)$ follows from:
    \begin{equation*}
        \tikzfig{tikz/func/F_comp2}
    \end{equation*}
    \begin{equation*}
        \tikzfig{tikz/func/F_comp3}
    \end{equation*}

    Which implies that controlled diagrams can fuse under $\lowerbox{\zspid}$.

    $\Control$ preserves the monoidal unit since $\mone_{\mathbf{Hilb_{sq}}} = \mone_{\mathbf{Hilb}} = \lowerbox{\emptydiag}$. $\Control$ is lax thanks to the following morphism: $\phi_{V, W}: \Control(V) \otimes \Control(W) \to \Control(V \otimes W)$:
    \begin{equation*}
        \phi_{V, W} = \tikzfig{tikz/func/F_lax_def}
    \end{equation*}

    $\phi$ is natural since for any $M_1: V \to V, M_2: W \to W$, we have:
    \begin{equation*}
        \tikzfig{tikz/func/F_lax_nat}
    \end{equation*}


    Finally, $\phi$ satisfies the coherence condition since for any $U, V, W$:
    \begin{equation*}
        \tikzfig{tikz/func/F_lax_ass}
    \end{equation*}


    
\end{proof}

\subsection{Multiple Control}

This section proves that controlling a controlled diagram gives the AND of the control wires. First we show how to represent the binary AND gate in the ZXW calculus.
\begin{prop}\label{prop:and}
    \begin{equation*}
        \tikzfig{tikz/func/and_def}
    \end{equation*}
\end{prop}


\begin{prop}
    \begin{equation*}
        \tikzfig{tikz/func/FF_statement}
    \end{equation*}
\end{prop}

\begin{proof}
    Plugging basis states:
    \begin{equation*}
        \tikzfig{tikz/func/FF1}
    \end{equation*}
    \begin{equation*}
        \tikzfig{tikz/func/FF2}
    \end{equation*}
\end{proof}

Since AND is a monoid, it is reasonable to expect that multiple control induces a monad. Unfortunately, this does not appear to go through. Although AND does define a natural transformation (in the relevant category), defining the unit of the monad necessitates defining the controlled diagram of an arbitrary matrix, which breaks functoriality. Nevertheless we prove that AND is a natural transformation.

\begin{prop}
    $\mu: \Control ^2 \to \Control$ is a natural transformation with components $\mu_V: \Control^2 V \to \Control V$ defined as follows:

    \begin{equation*}
        \tikzfig{tikz/func/mu_def}
    \end{equation*}
\end{prop}

\begin{proof}
    \begin{equation*}
        \tikzfig{tikz/func/mu_nat}
    \end{equation*}
    Due to the $Z-H$ bialgebra in the ZH calculus. 
\end{proof}

%\begin{prop}
%    $(F, \mu, \eta)$ defined a monad on $\mathbf{Hilb_j}$
%\end{prop}
%\begin{proof}
%    Firstly, $\mu$ is a natural transformation since for any $M: V \to V$:
%Similary, $\eta$ is a natural transformation since for any $M: V \to V$:
%    \begin{equation*}
%        \tikzfig{tikz/func/eta_nat}
%    \end{equation*}
% For the unitality coherence condition:
%    \begin{equation*}
%        \tikzfig{tikz/func/coh1}
%    \end{equation*}
%\end{proof}