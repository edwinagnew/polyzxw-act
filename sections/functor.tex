\section{The Control Higher-Order Map}\label{sec:ctrlmap}

\subsection{Functorial Boxes}

%The operation of turning a non-dimension-decreasing matrix to its controlled diagram can be made into a lax monoidal functor. Let $\mathbf{Hilb_{\leq}}$ be the subcategory of Hilbert spaces and non-dimension-decreasing linear transformations. Adding an additional horizontal wire to facilitate composition, $F: \mathbf{Hilb_{\leq}} \to \mathbf{Hilb}$ is defined as follows for arbitrary  $D \in Hom_{Hilb_{\leq}}(V, W)$.
We define the higher-order map $\Control$ which takes a square matrix $M: V \to V$ to its controlled square diagram $V \otimes \mathbb{C}^2 \to V \otimes \mathbb{C}^2$. In the functorial box notation of \cite{mellies2006functorial}, we write:
\begin{equation}
    \tikzfig{tikz/func/F_def_box}
\end{equation}

We prove in Appendix~\ref{sec:ctrlmapproofs} that composition of controlled operations in sequence and in parallel is well-behaved.

\begin{prop}\label{prop:ctrl_comp_h}
\begin{equation*}
	\tikzfig{tikz/func/ctrl_comp_hdef}
\end{equation*}\end{prop}

\begin{prop}\label{prop:ctrl_comp_v}
\begin{equation*}
	\tikzfig{tikz/func/ctrl_comp_vdef}
\end{equation*}
\end{prop}

\subsection{Multiple Control}

This section proves that controlling a controlled diagram gives the AND of the control wires. First we show how to represent the binary AND gate in the ZXW-calculus, by deMorgan's law of the diagram for binary OR given in \Cref{prop:vec_pnf}.
\begin{lemma}\label{lemma:and}
    \begin{equation*}
        \tikzfig{tikz/func/and_def}
    \end{equation*}
\end{lemma}

Multiple applications of $\Control$ ANDs the controls:
\begin{prop}\label{prop:FF}
    \begin{equation*}
        \tikzfig{tikz/func/FF_statement}
    \end{equation*}
\end{prop}