\section{Conclusion}
To conclude, we first introduced the higher-order map $\Control$ and showed it is a lax monoidal functor on all same-size square matrices. This enabled us to add functorial boxes to such ZXW diagrams. Moreover, we gave rewrite rules for interactions between controlled diagrams and all generators Z, X, W, and H.

We further proved completeness for all controlled $n$-partite states, which we showed form a commutative ring isomorphic to multilinear polynomials. Also, we showed that all controlled $n$-qubit square matrices form a non-commutative ring. Furthermore, we have completeness for plugging controlled states into the control wires of controlled diagrams, isomorphic to all multivariate polynomials over same-size square matrices, with application to factoring Hamiltonians. When the controls target mutually exclusive sectors, a rewrite rule can be applied to copy any controlled diagram, and thereby factor any Hamiltonian.

This work opens up connections between quantum circuit complexity and the far better understood algebraic complexity.
We have shown that every (controlled) state computes a polynomial; hence, we can interpret a universal fragment of the ZXW calculus as corresponding to arithmetic circuits. This generalises \cite{carette2023compositionality}, which found an algebraic interpretation of a certain fragment of ZW calculus. Reinterpreting quantum circuits as computing polynomials in relation to algebraic complexity was explored in the Master's thesis associated with this work~\cite{Agnew2023Masters}.
%Algebraic complexity deals with algebraic circuits and the polynomials they compute. Thus reinterpreting quantum circuits as computing polynomials rather than unitary matrices promises to offer a new perspective on quantum computation.

In another direction, we can apply completeness for polynomials isomorphic to controlled states to study entanglement. It can be easily shown diagrammatically that the polynomials corresponding to entangled (non-separable) states are exactly those that cannot be factored into irreducibles containing only variables corresponding to Alice's subsystem or only corresponding to Bob's subsystem. Since there are efficient algorithms for polynomial factorisation, this gives rise to a novel entanglement classification algorithm for pure states. Further developing this into a more refined algebraic theory of entanglement, building on the work in Ref.~\cite{Agnew2023Masters}, could offer further insights.

The natural next step is to derive extensions of our results for controlled qubit diagrams to qudits.
While the diagrams being controlled are over qudits, we can consider control in the qubit subspace, as done in the ZXW calculus completeness proof for any qudit dimension~\cite{poor2023completeness}.
A starting guess would be that qudit controlled states are isomorphic to polynomials $\mathbb{C}^{d-1}[x_1,...,x_n]/({x_1}^d,...,{x_n}^d)$ due to the Hopf law between Z and W.
Qudit multiple-control would likely have more complex structure than the qubit case here, considering the constructions for all prime-dimensional $d$-ary classical reversible gates built in Ref.~\cite{Roy2023quditzh}.

We would like to try sector-preserving channels~\cite{Vanrietvelde2021ctrlsector} and scoped effects~\cite{lindley2024scoped} as approaches to better formulate the monadic nature of multiple-control.
We are also curious about reconciling the interpretation of diagrammatic differentiation of our arithmetic polynomial circuits by the approach in Ref.~\cite{wilson2023diffpolycirc}, with that of quantum circuits and ZX diagrams in Refs.~\cite{toumi2021diagdiff, wang2022diffintzx, jeandel2024adddiffzx}.
Last but not least, these new semantics for quantum controlled states and matrices could be embedded categorically into a host functional programming language like in Ref.~\cite{rennela2020clctrllinlogic}, or translated to an equational theory for a quantum programming language like in Ref.~\cite{staton2015algqpl}.