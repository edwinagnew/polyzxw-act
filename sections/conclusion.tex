\section{Conclusion}
To conclude, we first introduced the higher-order map $\Control$ and showed it is a lax monoidal functor on all same-size square matrices. This enabled us to add functorial boxes to such ZXW diagrams. Moreover, we find useful rewrite rules for interactions between controlled diagrams and all generators Z, X, W, and H.

We further showed that all controlled $n$-partite states form a commutative ring isomorphic to multilinear polynomials, and all controlled $n$-qubit square matrices form a non-commutative ring. Plugging the former into the control wires of the latter, gives multivariate polynomials over same-size square matrices, such as Hamiltonians. When the controls target mutually exclusive sectors, a rewrite rule can be applied to copy any controlled diagram, and thereby factor any Hamiltonian.

The natural next step is to derive extensions of our results for controlled qubit diagrams to qudits.
While the diagrams being controlled are over qudits, we can consider control in the qubit subspace, as done in the ZXW calculus completeness proof for any qudit dimension~\cite{poor2023completeness}.
A starting guess would be that qudit controlled states are isomorphic to polynomials $\mathbb{C}^{d-1}[x_1,...,x_n]/({x_1}^d,...,{x_n}^d)$ due to the Hopf law between Z and W.
Qudit multiple-control would likely have more complex structure than the qubit case here, considering the constructions for all prime-dimensional $d$-ary classical reversible gates built in Ref.~\cite{Roy2023quditzh}.

We would like to try sector-preserving channels~\cite{Vanrietvelde2021ctrlsector} and scoped effects~\cite{lindley2024scoped} as approaches to better formulate the monadic nature of multiple-control.
We are also curious about reconciling the interpretation of diagrammatic differentiation of our arithmetic polynomial circuits by the approach in Ref.~\cite{wilson2023diffpolycirc}, with that of quantum circuits and ZX diagrams in Refs.~\cite{toumi2021diagdiff, wang2022diffintzx, jeandel2024adddiffzx}.
Last but not least, these new semantics for quantum controlled states and matrices could be embedded categorically into a host functional programming language like in Ref.~\cite{rennela2020clctrllinlogic}, or translated to an equational theory for a quantum programming language like in Ref.~\cite{staton2015algqpl}.