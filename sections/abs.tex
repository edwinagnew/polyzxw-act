\begin{abstract}
    Quantum control is an important logical primitive of quantum computing programs, and an important concept for graphical rewriting in quantum graphical calculi.
    In this work, we investigate the algebraic structure of \emph{controlled diagrams} in the ZXW-calculus --- diagrams extended with an additional qubit wire for triggering an operation on or off.
    By formalising these properties of quantum control, we enable powerful new graphical rewrite rules; these are applicable by virtue of control diagrams, therefore providing a higher level of abstraction agnostic to their implementation details.
    
    First, we prove that controlled square matrices form a ring, and thus admit expressive rewrite rules. We also show that controlled states form a ring, which is isomorphic to the ring of multilinear polynomials. Putting these together, we have completeness for polynomials over same-size square matrices, implying that these rules suffice to derive any factorisation of any Hamiltonian. Due to this result, it is unlikely that arbitrary quantum states can be efficiently rewritten to any diagram in what we define as the arithmetic fragment of the calculus, as this would imply $\mathsf{RP} = \mathsf{NQP}$.
\end{abstract}