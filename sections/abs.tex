\begin{abstract}
    An important concept for graphical rewriting in the ZXW calculus is a \emph{controlled diagram} --- a diagram extended with an additional input wire for triggering the operation on or off. Controlled diagrams have been applied in a number of areas including quantum chemistry, quantum machine learning, and photonics. In this work, we investigate the algebraic structure of controlled diagrams. First, we prove that the higher-order map which sends square matrices to their controlled square matrix is a lax monoidal functor. We then show that controlled matrices form a ring, yielding powerful rewrite rules for large classes of diagrams; we also show that controlled states form a ring, in this case isomorphic to the ring of multilinear polynomials. Augmenting the ZXW calculus with controlled square matrices as black-box generators, we prove completeness for polynomials over same-size square matrices. Through formalising the algebraic properties of quantum control, we give rise to powerful new graphical rewrite rules for black-box diagrams.
\end{abstract}