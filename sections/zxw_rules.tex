\begin{figure}[htbp]
  \centering
  \renewcommand{\arraystretch}{1.5}
  \begin{tabular}{|p{0.45\textwidth}|p{0.45\textwidth}|}
    % \hline
    % \multicolumn{2}{|c|}{\textbf{ZX Rules}} \\
    % \hline
    % \multicolumn{2}{|c|}{
    %   \begin{minipage}{\linewidth}
    %     \centering
    %     \begin{gather}
    %       \scalebox{0.9}{\tikzfig{tikz/axioms/s1}}
    %       \tag{S1}\label{rule:S1}
    %     \end{gather}
    %   \end{minipage}
    % } \\
    \hline
    \multicolumn{2}{|c|}{\textbf{ZX Rules}} \\
    \hline
    \begin{minipage}{\linewidth}
      \vspace{-1em}
      \begin{gather}
        \scalebox{0.9}{\tikzfig{tikz/axioms/s1}}
        \tag{S1}\label{rule:S1}\\
        \scalebox{0.9}{\tikzfig{tikz/axioms/s2}}
        \tag{S2}\label{rule:S2} \\
        \scalebox{0.9}{\tikzfig{tikz/axioms/k0copy}}
        \tag{K0}\label{rule:K0} \\
        \scalebox{0.9}{\tikzfig{tikz/axioms/zerotoreddit0}}
        \tag{Zer}\label{rule:Zer} \\
        \scalebox{0.9}{\tikzfig{tikz/axioms/rdotaemptydit0}}
        \tag{Ept}\label{rule:Ept}
      \end{gather}
    \end{minipage} &
    \noindent\colorbox{gray!20}{%
    \begin{minipage}{\linewidth}
      \vspace{-1em}
      \begin{gather}
        \scalebox{0.9}{\tikzfig{tikz/axioms/pimultiplecpdit}}
        \tag{K1}\label{rule:K1} \\
        \scalebox{0.9}{\tikzfig{tikz/axioms/k2adit}}
        \tag{K2}\label{rule:K2} \\
        \scalebox{0.9}{\tikzfig{tikz/axioms/h_id}}
        \tag{H}\label{rule:H} \\
        \scalebox{0.9}{\tikzfig{tikz/axioms/b2}}
        \tag{B2}\label{rule:B2}
      \end{gather}
    \end{minipage}} \\
    \hline
    \multicolumn{2}{|c|}{\textbf{ZW Rules}} \\
    \hline
    \begin{minipage}{\linewidth}
      \vspace{-1em}
      \begin{gather}
        \scalebox{0.9}{\tikzfig{tikz/axioms/phasecopydit}}
        \tag{Pcpy}\label{rule:Pcpy} \\
        \scalebox{0.9}{\tikzfig{tikz/axioms/additiondit}}
        \tag{Add}\label{rule:Add} \\
        \scalebox{0.9}{\tikzfig{tikz/axioms/w-bialgebra}}
        \tag{BZW}\label{rule:BZW}
      \end{gather}
    \end{minipage} &
    \begin{minipage}{\linewidth}
      \vspace{-1em}
      \begin{gather}
        \scalebox{0.9}{\tikzfig{tikz/axioms/associatedit}}
        \tag{Aso}\label{rule:Aso} \\
        \scalebox{0.9}{\tikzfig{tikz/axioms/w-w-algebra}}
        \tag{WW}\label{rule:WW}
      \end{gather}
    \end{minipage} \\
    \hline
    \multicolumn{2}{|c|}{\textbf{ZXW Rules}} \\
    \hline
    \begin{minipage}{\linewidth}
      \vspace{-1em}
      \begin{gather}
        \scalebox{0.9}{\tikzfig{tikz/axioms/triangleocopydit}}
        \tag{Bs0}\label{rule:Bs0} \\
        \scalebox{0.9}{\tikzfig{tikz/axioms/trialgebra}}
        \tag{TA}\label{rule:TA}
      \end{gather}
    \end{minipage} &
    \noindent\colorbox{gray!20}{%
    \begin{minipage}{\linewidth}
      \vspace{-1em}
      \begin{gather}
        \scalebox{0.9}{\tikzfig{tikz/axioms/trianglepicopydit2}}
        \tag{Bs1}\label{rule:Bs1}\\
        \scalebox{0.9}{\tikzfig{tikz/axioms/hadamard-decomposition2}}
        \tag{HD}\label{rule:HD}
      \end{gather}
    \end{minipage}} \\ 
    \hline
    \multicolumn{2}{|c|}{\textbf{Controlled Ring Rules}} \\
    \hline
    \begin{minipage}{\linewidth}
      \vspace{-1em}
      \begin{gather}
        \tikzfig{tikz/con/c_state0}
        \tag{CS0}\label{rule:cstate0} \\
        \scalebox{0.85}{\tikzfig{tikz/con/cs_copy}}
        \tag{CScpy}\label{rule:CScpy}
      \end{gather}
    \end{minipage} &
    \begin{minipage}{\linewidth}
      \vspace{-1em}
      \begin{gather}
        \scalebox{0.75}{\tikzfig{tikz/con/c_sq0}}
        \tag{CM0}\label{rule:c_sq0} \\
        \scalebox{0.75}{\tikzfig{tikz/con/csq_copy}}
        \tag{CMcpy}\label{rule:CMcpy}\\
        \scalebox{0.75}{\tikzfig{tikz/con/csq_add_comm_statement}}
        \tag{CMcom}\label{rule:CMcom}
      \end{gather}
    \end{minipage} \\
    \hline
  \end{tabular}
  \caption{These ZX, ZW, and ZXW Rules are altogether complete for qubit linear maps~\cite{poor2023completeness}, where $k \in \{0, 1\}$ and $a \in \mathbb{C}$.
  The white background ZX, ZW, and ZXW Rules here suffice for completeness of \emph{arithmetic diagrams} (Definition~\ref{def:arithmetic}), where \eqref{rule:TA} was used only to prove Lemma~\ref{lem:kill_quad}.
  Culminating in Theorem~\ref{thm:ctrl_pnf} of this work, we show that controlled states and controlled operators form rings. The above rules with white background achieve completeness for all operations over these rings, and we did not use the gray background rules in this work.}
  \label{fig:zxw_rules}
\end{figure}