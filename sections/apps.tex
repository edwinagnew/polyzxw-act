\section{Applications}

\subsection{Factorising Hamiltonians}
As an application, we leverage both our rewrite rules for arithmetic ZW diagrams, and for controlled diagrams, to \emph{factor} them.  For example, for same size square matrices $I, A, B$ and $a, b, c \in \mathbb{C}$:
\begin{gather*}
    \tikzfig{tikz/poly/matspoly}
\end{gather*}
Factoring Hamiltonians is important to optimise quantum algorithms for chemistry and physics simulations. However, previous graphical rewrites for factoring Hamiltonians had only been doable for Hamiltonians with concretely-specified matrix terms~\cite{shaikh2022sum}. This completeness result guarantees that for any Hamiltonian, even if its matrix terms are black-box, these graphical rewrite rules are capable of deriving any of its possible factorisations.


\subsection{The Control Higher-Order Map}\label{sec:ctrlmap}
%The operation of turning a non-dimension-decreasing matrix to its controlled diagram can be made into a lax monoidal functor. Let $\mathbf{Hilb_{\leq}}$ be the subcategory of Hilbert spaces and non-dimension-decreasing linear transformations. Adding an additional horizontal wire to facilitate composition, $F: \mathbf{Hilb_{\leq}} \to \mathbf{Hilb}$ is defined as follows for arbitrary  $D \in Hom_{Hilb_{\leq}}(V, W)$.
In quantum circuits, quantum control is realized through controlled gates.
In this section, we show that we can reason with controlled gates by a straightforward construction on top of our ring of controlled square matrices.
We define the higher-order map $\Control$ which takes a square matrix $M: V \to V$ to its controlled square diagram $V \otimes \mathbb{C}^2 \to V \otimes \mathbb{C}^2$. In the functorial box notation of \cite{mellies2006functorial}, we write:
\begin{equation}
    \tikzfig{tikz/func/F_def_box}
\end{equation}

We prove in Appendix~\ref{sec:ctrlmapproofs} that composition of controlled operations in sequence and in parallel is well-behaved.

\begin{prop}\label{prop:ctrl_comp_h}
\begin{equation*}
	\tikzfig{tikz/func/ctrl_comp_hdef}
\end{equation*}\end{prop}

\begin{prop}\label{prop:ctrl_comp_v}
\begin{equation*}
	\tikzfig{tikz/func/ctrl_comp_vdef}
\end{equation*}
\end{prop}

Furthermore, successive applications of $\Control$ recovers the standard notion of multiple-control, which computes the AND of the control qubits:
\begin{prop}\label{prop:FF}
    \begin{equation*}
        \tikzfig{tikz/func/FF_statement}
    \end{equation*}
\end{prop}