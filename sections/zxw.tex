\section{ZXW Calculus}


\subsection{Generators}

The (qubit) ZXW calculus is build from the following generators:

\begin{itemize}
  \item \textbf{Identity wire}: \begin{equation*}
    \lowerbox{\idwire[0.8]} ~:=~ \begin{bmatrix}
      1 & 0 & \\ 0 & 1
    \end{bmatrix}
  \end{equation*}
  \item \textbf{Swap}:\begin{equation*}
    \lowerbox[10]{\swap[0.4]} ~:=~ \begin{bmatrix}
      1 & 0 & 0 & 0 \\ 0 & 0 & 1 & 0 \\ 0 & 1 & 0 & 0 \\ 0 & 0 & 0 & 1
    \end{bmatrix}
  \end{equation*}
  \item \textbf{Z box}: \begin{equation*}
    \tikzfig{tikz/lemmas/zspid} ~:=~ |0^m\rangle\langle0^n| + e^{i\alpha}|1^m\rangle\langle1^n|, \alpha \in \mathbb{C}
  \end{equation*}
  \item \textbf{W node}: \begin{equation*}
    \raisebox{-7pt}{\wspid[0.7]} ~:=~ |00\rangle \braz + |01\rangle \brao + |10\rangle \brao
  \end{equation*}
  \item \textbf{H box}: \begin{equation*}\raisebox{-8pt}{\hgate} := \frac{1}{\sqrt{2}}\begin{bmatrix}1 & 1 \\ 1 & -1\end{bmatrix}\end{equation*}
\end{itemize}


\subsection{Rules}


\textbf{ZX Rules}:

\begin{gather}
  \tikzfig{tikz/axioms/s1}
  \tag{S1}\label{rule:S1}
\end{gather}
\begin{multicols}{2}
  \noindent
  \begin{gather*}
    \tikzfig{tikz/axioms/s2}
    \tag{S2}\label{rule:S2}
    \neweqline
    \tikzfig{tikz/axioms/rdotaemptydit0}
    \tag{Ept}\label{rule:Ept}
    \neweqline
    \tikzfig{tikz/axioms/b2}
    \tag{B2}\label{rule:B2}
    \neweqline
    \tikzfig{tikz/axioms/k0copy}
    \tag{K0}\label{rule:K0} 
  \end{gather*} \columnbreak
  \begin{gather*}
    \tikzfig{tikz/axioms/pimultiplecpdit}
    \tag{K1}\label{rule:K1}
    \neweqline
    \tikzfig{tikz/axioms/k2adit}
    \tag{K2}\label{rule:K2}
    \neweqline
    \tikzfig{tikz/axioms/zerotoreddit0}
    \tag{Zer}\label{rule:Zer}
    \neweqline
    \tikzfig{tikz/axioms/h_id}
    \tag{H}\label{rule:H} 
  \end{gather*}
\end{multicols}

Where $k \in \{0, 1\}$. 


\bigskip

\textbf{ZW Rules}:

\begin{multicols}{2}
  \noindent
  \begin{gather*}
    \tikzfig{tikz/axioms/phasecopydit}
    \tag{Pcy}\label{rule:Pcy}\neweqline
    \tikzfig{tikz/axioms/wsymetrydit}
    \tag{Sym}\label{rule:Sym}\neweqline
    \tikzfig{tikz/axioms/w-bialgebra}
    \tag{BZW}\label{rule:BZW}
    \end{gather*} \columnbreak
    \begin{gather*}
    \tikzfig{tikz/axioms/additiondit}
    \tag{ADD}\label{rule:ADD}\neweqline
    \tikzfig{tikz/axioms/associatedit}
    \tag{Aso}\label{rule:Aso}\neweqline
    \tikzfig{tikz/axioms/w-w-algebra}
    \tag{WW}\label{rule:WW}
  \end{gather*}
\end{multicols}


\bigskip

\textbf{ZXW Rules}:

\begin{multicols}{2}
  \noindent
  \begin{gather*}
    \tikzfig{tikz/axioms/triangleocopydit}
    \tag{Bs0}\label{rule:Bs0}\neweqline
    \tikzfig{tikz/axioms/trianglepicopydit2}
    \tag{Bsj}\label{rule:Bsj}\neweqline
    \tikzfig{tikz/axioms/trialgebra}
    \tag{TA}\label{rule:TA}\neweqline
    \tikzfig{tikz/axioms/hadamard-decomposition2}
    \tag{HD}\label{rule:HD}
    \end{gather*}
\end{multicols}


A number of basic lemmas are found in appendix \ref*{sec:applem}.
