\section{ZXW Calculus}

This section introduces the generators of the ZXW calculus. The ZXW calculus is a diagrammatic formalism for qudit computation, unifying the ZX and ZW calculi and synthesising their relative strenghts. The ZXW calculus consists of diagrams built from a small number of generators and equipped with a complete set of rewrite rules which enables all equalities between linear maps to be proven diagramatically.  Diagrams are to be read top to bottom or, in later sections, left to right.
\subsection{Generators}

The qubit ZXW calculus is built from the following generators:

\begin{gather*}
  \left\llbracket \quad \lowerbox{\idwire[0.5]} \;\;\; \right\rrbracket ~=~ \begin{bmatrix}1 & 0 \\ 0 & 1\end{bmatrix} \quad
  \left\llbracket \; \lowerbox[10]{\swap[0.4]} \; \right\rrbracket ~=~ \begin{bmatrix} 1 & 0 & 0 & 0 \\ 0 & 0 & 1 & 0 \\ 0 & 1 & 0 & 0 \\ 0 & 0 & 0 & 1\end{bmatrix} \quad
  \Big\llbracket \; \lowerbox[3]{\ccap} \; \Big\rrbracket ~=~ \begin{bmatrix} 1 \\ 0 \\ 0 \\ 1\end{bmatrix} \quad
  \left\llbracket \; \lowerbox[5]{\ccup} \; \right\rrbracket ~=~ \begin{bmatrix} 1 & 0 & 0 & 1 \end{bmatrix} \\
  \left\llbracket \;\; \tikzfig{tikz/defs/zspid} \;\; \right\rrbracket ~=~ |0^m\rangle\langle0^n| + c|1^m\rangle\langle1^n|, c \in \mathbb{C} \qquad
  \left\llbracket \;\; \raisebox{-10pt}{\wspid[0.7]} \;\; \right\rrbracket ~=~ |00\rangle \braz + |01\rangle \brao + |10\rangle \brao \\
  \left\llbracket \;\;\; \raisebox{-8pt}{\hgate} \;\; \right\rrbracket = \frac{1}{\sqrt{2}}\begin{bmatrix}1 & 1 \\ 1 & -1\end{bmatrix}
\end{gather*}

For simplicity, we introduce the following additional notation:

\begin{gather*}
  \tikzfig{tikz/defs/zcirc}\\
  \tikzfig{tikz/defs/xcirc}\qquad \qquad \qquad \qquad
  \raisebox{-10pt}{\coWs[0.7]} ~:=~ \tikzfig{tikz/defs/w_trans}
\end{gather*}


The complete rule set is given in Appendix \ref*{sec:apprules}. Several important lemmas are found in Appendix \ref*{sec:applem}.
