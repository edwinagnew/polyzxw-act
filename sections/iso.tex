\section{Isomorphism between the Ring of Controlled States and Multilinear Polynomials}\label{sec:iso}

Its been known since 2011 that $\lowerbox{\wspid}, \lowerbox{\zspid}$ can be used to add and multiply number states $\lowerbox{\numberstate}$, respectively \cite{coecke2011ghz}. In the previous section we saw that $\lowerbox{\wspid}, \lowerbox{\zspid}$ can moreover be used to copy controlled diagrams. In this section, we explain this connection by demonstrating that controlled states are in fact isomorphic to multilinear polynomials. This being a bijection is a well-known folklore result in the study of entangled states, but to the best of our inquiries we are not aware of a proof. More generally, Ref.~\cite{wilson2023diffpolycirc} presented Cartesian Distributive Categories exemplified by polynomial circuits, which are isomorphic to polynomials over arbitrary commutative semirings or rings; their proof is non-constructive, giving explicit proof only for the case of Boolean circuits~\cite{wilson2021revderbool}.

Firstly, we describe how to interpret certain ZXW diagrams as polynomials. Consider the diagrams:
\begin{equation*}
    \tikzfig{tikz/poly/eg1}
\end{equation*}

If we treat the bottom wires as an indeterminate $x$, we can read these bottom-up as computing $x - 1$ and $2x + 3$, respectively. Moreover, since these diagrams are both controlled states, they can be added together,  yield a diagram resembling $3x + 2$:
\begin{equation*}
    \tikzfig{tikz/poly/eg4}
\end{equation*}

When trying to multiply these diagrams, rather than getting $(x-1)(2x+3) = 2x^2 + x - 3$, we instead get $x - 3$.
\begin{equation*}
    \tikzfig{tikz/poly/eg5}
\end{equation*}

The reason for the missing $2x^2$ term is that Lemma \ref{eq:kill_quad} implies $x^2 = 0$. Other than that, controlled state arithmetic appears to faithfully reflect polynomial arithmetic. To help formalise this correspondence, we introduce the following definition.

\begin{definition}
    A ZXW diagram with a single input on top is \textbf{arithmetic} if it contains only  $\lowerbox{\idwire}$, $\lowerbox{\swap}$ wires, $\lowerbox[10]{\wspids}$, $\lowerbox[10]{\zspids}$, $\lowerbox{\coWs}$ nodes and $\lowerbox{\numberstate}$ boxes.
\end{definition}
\begin{remark}
    This fragment of the ZXW-calculus defines a subcategory; adding $\lowerbox{\redzero}$, this is an instance of a Cartesian Distributive Category as defined in Ref.~\cite{wilson2023diffpolycirc}.
\end{remark}

To interpret an arithmetic ZXW diagram as an arithmetic expression, read $\lowerbox{\wspid}$ as $+$, $\lowerbox{\zspid}$ as $\times$, $\lowerbox{\numberstate}$ as the number $a$, $\lowerbox{\coW}$ as fanout and output/bottom wires as variables $x_1, ..., x_n$ numbered from left to right. The following lemma establishes that all arithmetic diagrams are controlled states:
\begin{lemma}
    For any arithmetic diagram $A$, \begin{equation}\label{eq:arith_cs}
        \tikzfig{tikz/poly/arith0_statement}
    \end{equation}
\end{lemma}
\begin{proof}
    By definition, other than wires $A$ contains only $\raisebox{-10pt}{\wspids}$, $\raisebox{-10pt}{\zspids}$, $\raisebox{-5pt}{\coWs}$, and $\raisebox{-3pt}{\numberstate}$. All $\raisebox{-3pt}{\numberstate}$'s can be removed with (\ref{rule:Ept}). Meanwhile all the spiders copy $\raisebox{-5pt}{\redzero}$ due to (\ref{rule:Bs0}, \ref{rule:K0}, \ref{eq:wid}) respectively.
\end{proof}

Just as it is typical to represent a polynomial in normal form as a sum of products, it is possible to rewrite every arithmetic diagram into a normal form as a single $\lowerbox[10]{\wspids}$, followed by a layer of $\lowerbox[10]{\zspids}$, followed by a layer of $\lowerbox[2]{\numberstate}, \lowerbox[2]{\coWs}$. 

\begin{definition}
    An $n$-output arithmetic diagram is said to be written in \textbf{polynormal form} (PNF) if it is of the form:
    \begin{equation}\label{eq:pnf}
        \tikzfig{tikz/poly/pnf}
    \end{equation}

    The $i$th coefficient $a_i$ is connected to the $k$th $\lowerbox{\coWs}$ iff the $k$th bit in the binary expansion of $i$ is 1. 
\end{definition}

This normal form is familiar from completeness of all linear maps for qubits~\cite{Hadzihasanovic2018zwzxcomplete} and for qudits~\cite{poor2023completeness}. The reason we introduce the definition of a PNF is that it is an arithmetic diagram and therefore has a more immediate arithmetic interpretation. The reason for the specific connectivity condition is that it enables a PNF to directly represent its own matrix.

\begin{prop}\label{prop:vec_pnf}
The diagram in equation (\ref{eq:pnf}) equals the matrix
    $\begin{bmatrix}
            1 &  a_0 \\ 0 & a_1 \\ ... & ... \\ 0 & a_{2^n-1}
        \end{bmatrix}$
\end{prop} 

\begin{proof}
    See Appendix~\ref*{sec:appiso}.
\end{proof}

Thus, every controlled state can be represented as at least one arithmetic diagram (namely, its PNF). Moreover, we now show that any other arithmetic diagram can always be rewritten to its PNF.


\begin{thm}\label{thm:uni_pnf}
    All arithmetic diagrams can be rewritten into PNF through application of ZXW rules. Therefore, those ZXW-calculus rules applied suffice for completeness for the arithmetic fragment of the ZXW-calculus.
\end{thm}

\begin{proof}
    An algorithm to rewrite any arithmetic diagram to PNF is given in Appendix~\ref*{sec:appiso}.
\end{proof}

\subsection{Isomorphism}

At last we can prove the isomorphism. Throughout we shall let $\polyring$ denote the multilinear ring $\mathbb{C}[x_1, ..., x_{n}]/(x_1^2, ..., x_{n}^2)$. Recall that $\tilde{S_n}$ is the ring of controlled states with $n$ outputs. 


\begin{thm}\label{thm:iso}
    There is an isomorphism $\polyring \simeq \tilde{S_n}$
\end{thm}

First, we shall define the map $\phi_n: \polyring \to \csring$ before proving it induces an isomorphism. $\phi_n$ is defined to map an arbitrary polynomial $p(x_1, ..., x_n) = a_0 + a_1x_n + ... + a_{2^n-1}x_1x_2...x_n$ to the PNF in equation (\ref{eq:pnf}).


Some important special cases are mapping scalars $a \in \mathbb{C}$ and indeterminates $x_i$:
 \begin{equation*}
        \tikzfig{tikz/poly/homproof/homdef1} \quad \tikzfig{tikz/poly/homproof/homdef2}
\end{equation*}
   
That $\phi_n$ is a bijection relies on proposition \ref{prop:vec_pnf}. The full proof is found in Appendix~\ref*{sec:appiso}.
